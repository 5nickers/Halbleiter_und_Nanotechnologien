\input{header.tex}


\begin{document}

\maketitle

Dieser Text ist unter dieser \href{http://creativecommons.org/licenses/by-nc-sa/4.0/}{Creative Commons} Lizenz veröffentlicht.

\textcolor{red}{Ich erhebe keinen Anspruch auf Vollständigkeit oder Richtigkeit. Falls ihr Fehler findet oder etwas fehlt, dann meldet euch bitte über den Emailkontakt.}

\tableofcontents


\newpage



\section{Aufgabe 1}

siehe auf die Zeichnung in Aufgabe 2.


\section{Aufgabe 2}

Die Lösung wird nur skizziert, weil die Lösung schon im Skript steht.

Wir starten mit der Schrödinger Gleichung:

\begin{align*}
0 &= \frac{\p^2 \Psi}{\p x^2} + \frac{2m}{\hbar} \cdot \left( E - V(x) \right) \cdot \Psi(x)
\intertext{Dabei ist $\Psi(x)$}
\Psi(x) &= u(x) \cdot e^{ikx}
\intertext{Dann gilt in den Bereichen I und II:}
\Psi_I &= u_1(x) \cdot e^{ik_1x} \\
\Psi_{II} &= u_2(x) \cdot e^{ik_2x} 
\intertext{Wir setzen in die Schrödinger Gleichung ein:}
0 &= u_1'' + 2ik_1 u_1' - \left( k_1^2 - \alpha^2 \right) \qquad \text{mit} \qquad \alpha^2 = \frac{2mE}{\hbar} \\
0 &= u_2'' + 2ik_2 u_1' - \left( k_2^2 - \alpha^2 \right) 
\intertext{Daraus ergeben sich die allgemeinen Lösungen:}
u_1(x) &= A \cdot e^{i \left( \alpha - k_1 \right)x} + B \cdot e^{i \left( \alpha - k_1 \right)x} \qquad \text{mit} \qquad \beta^2 = \alpha^2 - \frac{2mV_0}{\hbar} \\
u_2(x) &= C \cdot e^{i \left( \alpha - k_2 \right)x} + D \cdot e^{i \left( \alpha - k_2 \right)x}
\intertext{Aus der Stetigkeit der Übergang ergeben sich folgende Randbedingungen:}
u_1(0) &= u_2(0) \qquad u_1(a) = u_2(-b) \\
u_1'(0) &= u_2'(0) \qquad u_1'(a) = u_2'(-b)
\intertext{Aus der Vorlesunf wissen wir, das mir dieses Problem mit einer Matrix lösen müssen:}
\underbrace{
\left[
\begin{matrix}
...&...  &...  &...  \\ 
&  &  &  \\ 
&  &  &  \\ 
&  &  & 
\end{matrix} 
\right]
}_{\text{m}}
\cdot
\left(
\begin{matrix}
A \\ 
B \\ 
C \\ 
D
\end{matrix} 
\right)
= 
0
\end{align*}

Damit das erfüllt ist gelten $\det(m) = 0$.


\section{Aufgabe 2}

Im Prinzip ist das nur intelligentes Umformen:

\begin{align*}
E &= \frac{\hbar^2 k^2}{2m} \qquad \Leftrightarrow \qquad k = \sqrt{\frac{2mE}{\hbar^2}}
\intertext{Damit wir $\d E$ erhalten müssen wir $E$ nach $k$ ableiten:}
\frac{\d E}{\d k} &= \frac{\hbar^2 k}{2m} \\
\Leftrightarrow \d k &= \frac{\d E \cdot m}{\hbar k} = \frac{\d E \cdot m \hbar}{\hbar^2 \cdot \sqrt{2mE}} = \frac{1}{\hbar} \cdot \sqrt{\frac{m}{2E}} \d E
\intertext{Wir können nun $\d k$ ersetzen:}
D(k) \d k &= \frac{\pi 2 m E \cdot \sqrt{m}}{\pi^3 \hbar^3 \cdot \sqrt{2E}} \d E = \frac{1}{\pi^2 \hbar^3} \cdot \sqrt{2E} \cdot \sqrt{m}^3 \d E
\intertext{Nun haben wir:}
D(E) \d E &= \frac{1}{\pi^2 \hbar^3} \cdot \half \cdot \left( 2m \right)^{3/2} \cdot \sqrt{E} \d E
\intertext{Mit $\hbar = \frac{h}{2 \pi}$ erhalten wir:}
&= \frac{4 \pi \cdot \left( 2m \right)^{3/2)} \cdot \sqrt{E}}{h^3}
\end{align*}


\section{Aufgabe 4}


\begin{figure}[h]
	\centering
	\includegraphics[scale=0.15]{A4_1.jpg}
\end{figure}

Wir müssen hier die Zustandsdichte über ein Intervall integrieren:

\begin{align*}
n &= \int_{0}^{1} D(E) \d E = \int_{0}^{1} \frac{4 \pi \cdot \left( 2m \right)^{3/2)} \cdot \sqrt{E}}{h^3} \\
&= \left. \frac{4 \pi \cdot \left( 2m \right)^{3/2}}{h^3} \frac{2}{3} \cdot E^{3/2}  \right|_0^1 \\
&= \frac{4\pi \left( 29,11 \cdot 10^{-31} \right)^{3/2}}{6,625 \cdot 10^{-34}} \cdot \frac{2}{3} \cdot \left( 1,6 \cdot 10^{-19} \right)^{3/2} = \unit[4,5 \cdot 10^{27}]{m^{-3}}
\end{align*}

Das sind dann Zustände pro Kubikmeter



\section{Aufgabe 5}

\subsection*{1D}

Wir definieren uns zunächst ein $\d z$:

\begin{align*}
\d z &= \frac{\text{Strecke}}{\text{Strecke für einen Zustand}} = \frac{\d k_x}{\frac{\pi}{a}} = \frac{a \d k_x}{\pi}
\intertext{Nun normieren wir $\d z$:}
\d z' &= \frac{\d z}{a} = \frac{\d k_x}{\pi}
\intertext{Nun bestimmen wir $k_x$:}
\d E &= \frac{\hbar^2 k_x}{m} \d k \\
\Leftrightarrow \d k &= \frac{m}{\hbar^2 k_x} \d E \\
\Leftrightarrow k_x &= \sqrt{\frac{2mE}{\hbar^2}}
\intertext{Nun können wir in die ursprüngliche Gleichung einsetzen:}
\d z' &= \frac{m \hbar }{\pi \hbar^2 \cdot \sqrt{2mE}} \d E = \frac{\sqrt{m}}{\pi \hbar \cdot \sqrt{2}} \cdot \frac{1}{\sqrt{E}} \d E
\end{align*}


\subsection*{2D}

Wir definieren uns wieder das $\d z$:

\begin{align*}
\d z &= \frac{\overbrace{2 \pi k \d k}^{\text{Fläche eines Kreisrings}}}{\left( \frac{\pi}{a} \right)} = \frac{2 \pi k a^2 \d k}{\pi^2}
\intertext{Wir normieren wieder:}
\d z' &= \frac{\d z}{a^2} = \frac{2 \pi k \d k}{\pi}
\intertext{Wir verwenden das $\d k$ aus dem vorigen Teil:}
&= \frac{2 \pi}{\pi^2} \cdot \sqrt{\frac{2mE}{\hbar^2}} \cdot \frac{m}{\hbar^2 \cdot k} \d E = \frac{2m}{\pi \hbar^2} \d E
\end{align*}

Wir sehen, das im zweidimensionalen Fall die Zustandsdichte konstant ist.


\section{Aufgabe 6}

\subsection*{a)}

Wir wissen das für die Energie der Elektonen $E = 3kT + E_F$ gilt. Wir können dies nun in die Zustandswahrscheinlichkeit einsetzen:

\begin{align*}
f_h(E) &= \frac{1}{1 + e^{\frac{E - E_F}{kT}}} = \frac{1}{1 + e^{\frac{3kT}{kT}}} = \frac{1}{1 + e^3} \\
&= 0,0474 = \unit[4,47]{\%}
\end{align*}

\subsection*{b)}

Hier sieht das ganze rech ähnlich aus:

\begin{align*}
f_h(E) 1 - f(E) = \frac{1}{1 + e^{- \frac{E - E_F}{kT}}} = 0,0474 = \unit[4,74]{\%}
\end{align*}



schematisch sieht das ungefähr so aus:

\begin{figure}[h]
	\centering
	\includegraphics[scale=0.1]{A6_1.jpg}
\end{figure}


\section{Aufgabe 7}

Wir nehmen die Fermiverteilung von oben und setzen sie mit der Bolzmanverteilung gleich:

\begin{align*}
f(E) &= \frac{1}{1 + e^{\frac{E - E_F}{kT}}} = \frac{1}{1 + e^x} \qquad \text{mit} \qquad x = \frac{E - E_F}{kT} \\
\intertext{Wir setzen gleich:}
f_B(E) &= e^{- \frac{E - E_F}{kT}} = e^{-x} \\
\intertext{Dabei solle nur $\unit[5]{\%}$ Fehler gemacht werden:}
\frac{5}{100} &= \frac{f_B(E) - f(E)}{f(E)} = \frac{e^{-x} + \frac{1}{1 + e^x}}{\frac{1}{1 + e^x}} = \frac{\frac{e^{-x} + e^0 - 1}{1 + e^x}}{\frac{1}{1 + e^x}} \\
\Leftrightarrow e^{-x} &= \frac{5}{100} \\
\Leftrightarrow x &= - \ln\left( \frac{5}{100} \right) \approx 3 \\
\Rightarrow \frac{E - E_F}{kT} &= 3 \qquad \rightarrow \qquad \Delta E = 3kT
\end{align*}


Verbildlicht sieht das ca so aus:

\begin{figure}[h]
	\centering
	\includegraphics[scale=0.1]{A7_1.jpg}
\end{figure}


\newpage

\section{Aufgabe 8}

Alle Informationen sind in diesem Bild zusammengefasst:

\begin{figure}[h]
	\centering
	\includegraphics[scale=0.16]{A8_1.jpg}
\end{figure}











\end{document}