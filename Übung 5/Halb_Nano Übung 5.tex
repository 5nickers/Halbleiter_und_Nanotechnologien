\input{header.tex}


\begin{document}

\maketitle

Dieser Text ist unter dieser \href{http://creativecommons.org/licenses/by-nc-sa/4.0/}{Creative Commons} Lizenz veröffentlicht.

\textcolor{red}{Ich erhebe keinen Anspruch auf Vollständigkeit oder Richtigkeit. Falls ihr Fehler findet oder etwas fehlt, dann meldet euch bitte über den Emailkontakt.}

\tableofcontents


\newpage



\section{Aufgabe 1}

siehe auf die Zeichnung in Aufgabe 2.


\section{Aufgabe 2}

Die Lösung wird nur skizziert, weil die Lösung schon im Skript steht.

Wir starten mit der Schrödinger Gleichung:

\begin{align*}
0 &= \frac{\p^2 \Psi}{\p x^2} + \frac{2m}{\hbar} \cdot \left( E - V(x) \right) \cdot \Psi(x)
\intertext{Dabei ist $\Psi(x)$}
\Psi(x) &= u(x) \cdot e^{ikx}
\intertext{Dann gilt in den Bereichen I und II:}
\Psi_I &= u_1(x) \cdot e^{ik_1x} \\
\Psi_{II} &= u_2(x) \cdot e^{ik_2x} 
\intertext{Wir setzen in die Schrödinger Gleichung ein:}
0 &= u_1'' + 2ik_1 u_1' - \left( k_1^2 - \alpha^2 \right) \qquad \text{mit} \qquad \alpha^2 = \frac{2mE}{\hbar} \\
0 &= u_2'' + 2ik_2 u_1' - \left( k_2^2 - \alpha^2 \right) 
\intertext{Daraus ergeben sich die allgemeinen Lösungen:}
u_1(x) &= A \cdot e^{i \left( \alpha - k_1 \right)x} + B \cdot e^{i \left( \alpha - k_1 \right)x} \qquad \text{mit} \qquad \beta^2 = \alpha^2 - \frac{2mV_0}{\hbar} \\
u_2(x) &= C \cdot e^{i \left( \alpha - k_2 \right)x} + D \cdot e^{i \left( \alpha - k_2 \right)x}
\intertext{Aus der Stetigkeit der Übergang ergeben sich folgende Randbedingungen:}
u_1(0) &= u_2(0) \qquad u_1(a) = u_2(-b) \\
u_1'(0) &= u_2'(0) \qquad u_1'(a) = u_2'(-b)
\intertext{Aus der Vorlesunf wissen wir, das mir dieses Problem mit einer Matrix lösen müssen:}
\underbrace{
\left[
\begin{matrix}
...&...  &...  &...  \\ 
&  &  &  \\ 
&  &  &  \\ 
&  &  & 
\end{matrix} 
\right]
}_{\text{m}}
\cdot
\left(
\begin{matrix}
A \\ 
B \\ 
C \\ 
D
\end{matrix} 
\right)
= 
0
\end{align*}

Damit das erfüllt ist gelten $\det(m) = 0$.




































\end{document}