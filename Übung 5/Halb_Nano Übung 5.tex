\documentclass[11pt]{scrartcl}
\usepackage[T1]{fontenc}
\usepackage[a4paper, left=3cm, right=2cm, top=2cm, bottom=2cm]{geometry}
\usepackage[activate]{pdfcprot}
\usepackage[ngerman]{babel}
\usepackage[parfill]{parskip}
\usepackage[utf8]{inputenc}
\usepackage{kurier}
\usepackage{amsmath}
\usepackage{amssymb}
\usepackage{xcolor}
\usepackage{epstopdf}
\usepackage{txfonts}
\usepackage{fancyhdr}
\usepackage{graphicx}
\usepackage{prettyref}
\usepackage{hyperref}
\usepackage{eurosym}
\usepackage{setspace}
\usepackage{units}
\usepackage{eso-pic,graphicx}
\usepackage{icomma}
\usepackage{pdfpages}

\definecolor{darkblue}{rgb}{0,0,.5}
\hypersetup{pdftex=true, colorlinks=true, breaklinks=false, linkcolor=black, menucolor=black, pagecolor=black, urlcolor=darkblue}



\setlength{\columnsep}{2cm}


\newcommand{\arcsinh}{\mathrm{arcsinh}}
\newcommand{\asinh}{\mathrm{arcsinh}}
\newcommand{\ergebnis}{\textcolor{red}{\mathrm{Ergebnis}}}
\newcommand{\fehlt}{\textcolor{red}{Hier fehlen noch Inhalte.}}
\newcommand{\betanotice}{\textcolor{red}{Diese Aufgaben sind noch nicht in der Übung kontrolliert worden. Es sind lediglich meine Überlegungen und Lösungsansätze zu den Aufgaben. Es können Fehler enthalten sein!!! Das Dokument wird fortwährend aktualisiert und erst wenn das \textcolor{black}{beta} aus dem Dateinamen verschwindet ist es endgültig.}}
\newcommand{\half}{\frac{1}{2}}
\renewcommand{\d}{\, \mathrm d}
\newcommand{\punkte}{\textcolor{white}{xxxxx}}
\newcommand{\p}{\, \partial}
\newcommand{\dd}[1]{\item[#1] \hfill \\}

\renewcommand{\familydefault}{\sfdefault}
\renewcommand\thesection{}
\renewcommand\thesubsection{}
\renewcommand\thesubsubsection{}


\newcommand{\themodul}{Halbleiter und Nanotechnologie}
\newcommand{\thetutor}{Prof. Förster}
\newcommand{\theuebung}{Übung 3}

\pagestyle{fancy}
\fancyhead[L]{\footnotesize{C. Hansen}}
\chead{\thepage}
\rhead{}
\lfoot{}
\cfoot{}
\rfoot{}

\title{\themodul{}, \theuebung{}, \thetutor}


\author{Christoph Hansen \\ {\small \href{mailto:chris@university-material.de}{chris@university-material.de}} }

\date{}


\begin{document}

\maketitle

Dieser Text ist unter dieser \href{http://creativecommons.org/licenses/by-nc-sa/4.0/}{Creative Commons} Lizenz veröffentlicht.

\textcolor{red}{Ich erhebe keinen Anspruch auf Vollständigkeit oder Richtigkeit. Falls ihr Fehler findet oder etwas fehlt, dann meldet euch bitte über den Emailkontakt.}

\tableofcontents


\newpage



\section{Aufgabe 1}

siehe auf die Zeichnung in Aufgabe 2.


\section{Aufgabe 2}

Die Lösung wird nur skizziert, weil die Lösung schon im Skript steht.

Wir starten mit der Schrödinger Gleichung:

\begin{align*}
0 &= \frac{\p^2 \Psi}{\p x^2} + \frac{2m}{\hbar} \cdot \left( E - V(x) \right) \cdot \Psi(x)
\intertext{Dabei ist $\Psi(x)$}
\Psi(x) &= u(x) \cdot e^{ikx}
\intertext{Dann gilt in den Bereichen I und II:}
\Psi_I &= u_1(x) \cdot e^{ik_1x} \\
\Psi_{II} &= u_2(x) \cdot e^{ik_2x} 
\intertext{Wir setzen in die Schrödinger Gleichung ein:}
0 &= u_1'' + 2ik_1 u_1' - \left( k_1^2 - \alpha^2 \right) \qquad \text{mit} \qquad \alpha^2 = \frac{2mE}{\hbar} \\
0 &= u_2'' + 2ik_2 u_1' - \left( k_2^2 - \alpha^2 \right) 
\intertext{Daraus ergeben sich die allgemeinen Lösungen:}
u_1(x) &= A \cdot e^{i \left( \alpha - k_1 \right)x} + B \cdot e^{i \left( \alpha - k_1 \right)x} \qquad \text{mit} \qquad \beta^2 = \alpha^2 - \frac{2mV_0}{\hbar} \\
u_2(x) &= C \cdot e^{i \left( \alpha - k_2 \right)x} + D \cdot e^{i \left( \alpha - k_2 \right)x}
\intertext{Aus der Stetigkeit der Übergang ergeben sich folgende Randbedingungen:}
u_1(0) &= u_2(0) \qquad u_1(a) = u_2(-b) \\
u_1'(0) &= u_2'(0) \qquad u_1'(a) = u_2'(-b)
\intertext{Aus der Vorlesunf wissen wir, das mir dieses Problem mit einer Matrix lösen müssen:}
\underbrace{
\left[
\begin{matrix}
...&...  &...  &...  \\ 
&  &  &  \\ 
&  &  &  \\ 
&  &  & 
\end{matrix} 
\right]
}_{\text{m}}
\cdot
\left(
\begin{matrix}
A \\ 
B \\ 
C \\ 
D
\end{matrix} 
\right)
= 
0
\end{align*}

Damit das erfüllt ist gelten $\det(m) = 0$.




































\end{document}