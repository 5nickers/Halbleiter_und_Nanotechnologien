\input{header.tex}


\begin{document}

\maketitle

Dieser Text ist unter dieser \href{http://creativecommons.org/licenses/by-nc-sa/4.0/}{Creative Commons} Lizenz veröffentlicht.

\textcolor{red}{Ich erhebe keinen Anspruch auf Vollständigkeit oder Richtigkeit. Falls ihr Fehler findet oder etwas fehlt, dann meldet euch bitte über den Emailkontakt.}

\tableofcontents


\newpage



\section{Aufgabe 1}

\subsection*{a)}

Der Strom aus dem Behälter ist:

\begin{align*}
j_N &= \frac{N}{V} \cdot \frac{<v>}{4} = \frac{\p N}{\p t} \cdot \frac{1}{A} \\
\Leftrightarrow N(t + \d t) &= N(t) - j_{N(t)} \cdot A \cdot \d t \\
\Leftrightarrow n(t + \d t) &= n(t) - \frac{j_{N(t)} \cdot A \cdot \d t}{V} \\
\Leftrightarrow \frac{n(t + \d t) - n(t)}{\d t} &= - \frac{j_{N(t) \cdot A}}{V} = \frac{-n(t) \cdot <v> \cdot A}{V \cdot 4} \\
\underset{\d t \rightarrow 0}{lim} \qquad \frac{\dot{n}}{n} &= \frac{<v> \cdot A}{4V} := \frac{1}{\tau} \\
\hfill \\
\Rightarrow n(t) &= n_0 \cdot e^{-\frac{t}{\tau}}
\intertext{Wir rechnen ein Beispiel mit $N_2$ Gas bei $T = \unit[300]{K}$ und $P = \unit[1]{mbar}$, dabei ist $<v> = \unit[426]{m/s}$.}
\tau &= \frac{4 \cdot 1}{426 \cdot \frac{\left( 10^{-3} \right)^2 \pi}{4}} = \unit[1,2 \cdot 10^4]{s}
\intertext{Dieses $\tau$ nutzen wir jetzt für die Bestimmung des $\unit[50]{\%}$ Wertes:}
\half n_0 &= n_0 \cdot e^{- \frac{t}{\tau}} \\
\Leftrightarrow t &= -\tau \ln\left( \half \right) = \unit[8,3 \cdot 10^3]{s}
\end{align*}


\section{Aufgabe 2}

\subsection*{a)}

Wir berechnen zunächst den Clausing Faktor:

\begin{align*}
K'' &= \frac{15 \cdot \frac{1}{0,05} + 12 \cdot \left( \frac{1}{0,05} \right)^2}{20 + 38 \cdot \frac{1}{0,05} + 12 \cdot \left( \frac{1}{0,05} \right)^2} = 0,91
\intertext{Der Leitwert ist dann:}
C &= 0,91 \cdot 1,2 \cdot \sqrt{\frac{300}{28 \cdot 10^{-3}}} \cdot \frac{\left( 50 \cdot 10^{-3} \right)^2}{1} = \unit[0,0142]{m^3/s}
\end{align*}

Die andere Teilaufgaben gehen genauso, deshalb spare ich mir die!





\end{document}