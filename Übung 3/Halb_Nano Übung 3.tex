\documentclass[11pt]{scrartcl}
\usepackage[T1]{fontenc}
\usepackage[a4paper, left=3cm, right=2cm, top=2cm, bottom=2cm]{geometry}
\usepackage[activate]{pdfcprot}
\usepackage[ngerman]{babel}
\usepackage[parfill]{parskip}
\usepackage[utf8]{inputenc}
\usepackage{kurier}
\usepackage{amsmath}
\usepackage{amssymb}
\usepackage{xcolor}
\usepackage{epstopdf}
\usepackage{txfonts}
\usepackage{fancyhdr}
\usepackage{graphicx}
\usepackage{prettyref}
\usepackage{hyperref}
\usepackage{eurosym}
\usepackage{setspace}
\usepackage{units}
\usepackage{eso-pic,graphicx}
\usepackage{icomma}
\usepackage{pdfpages}

\definecolor{darkblue}{rgb}{0,0,.5}
\hypersetup{pdftex=true, colorlinks=true, breaklinks=false, linkcolor=black, menucolor=black, pagecolor=black, urlcolor=darkblue}



\setlength{\columnsep}{2cm}


\newcommand{\arcsinh}{\mathrm{arcsinh}}
\newcommand{\asinh}{\mathrm{arcsinh}}
\newcommand{\ergebnis}{\textcolor{red}{\mathrm{Ergebnis}}}
\newcommand{\fehlt}{\textcolor{red}{Hier fehlen noch Inhalte.}}
\newcommand{\betanotice}{\textcolor{red}{Diese Aufgaben sind noch nicht in der Übung kontrolliert worden. Es sind lediglich meine Überlegungen und Lösungsansätze zu den Aufgaben. Es können Fehler enthalten sein!!! Das Dokument wird fortwährend aktualisiert und erst wenn das \textcolor{black}{beta} aus dem Dateinamen verschwindet ist es endgültig.}}
\newcommand{\half}{\frac{1}{2}}
\renewcommand{\d}{\, \mathrm d}
\newcommand{\punkte}{\textcolor{white}{xxxxx}}
\newcommand{\p}{\, \partial}
\newcommand{\dd}[1]{\item[#1] \hfill \\}

\renewcommand{\familydefault}{\sfdefault}
\renewcommand\thesection{}
\renewcommand\thesubsection{}
\renewcommand\thesubsubsection{}


\newcommand{\themodul}{Halbleiter und Nanotechnologie}
\newcommand{\thetutor}{Prof. Förster}
\newcommand{\theuebung}{Übung 3}

\pagestyle{fancy}
\fancyhead[L]{\footnotesize{C. Hansen}}
\chead{\thepage}
\rhead{}
\lfoot{}
\cfoot{}
\rfoot{}

\title{\themodul{}, \theuebung{}, \thetutor}


\author{Christoph Hansen \\ {\small \href{mailto:chris@university-material.de}{chris@university-material.de}} }

\date{}


\begin{document}

\maketitle

Dieser Text ist unter dieser \href{http://creativecommons.org/licenses/by-nc-sa/4.0/}{Creative Commons} Lizenz veröffentlicht.

\textcolor{red}{Ich erhebe keinen Anspruch auf Vollständigkeit oder Richtigkeit. Falls ihr Fehler findet oder etwas fehlt, dann meldet euch bitte über den Emailkontakt.}

\tableofcontents


\newpage



\section{Aufgabe 1}

\subsection*{a)}

Der Strom aus dem Behälter ist:

\begin{align*}
j_N &= \frac{N}{V} \cdot \frac{<v>}{4} = \frac{\p N}{\p t} \cdot \frac{1}{A} \\
\Leftrightarrow N(t + \d t) &= N(t) - j_{N(t)} \cdot A \cdot \d t \\
\Leftrightarrow n(t + \d t) &= n(t) - \frac{j_{N(t)} \cdot A \cdot \d t}{V} \\
\Leftrightarrow \frac{n(t + \d t) - n(t)}{\d t} &= - \frac{j_{N(t) \cdot A}}{V} = \frac{-n(t) \cdot <v> \cdot A}{V \cdot 4} \\
\underset{\d t \rightarrow 0}{lim} \qquad \frac{\dot{n}}{n} &= \frac{<v> \cdot A}{4V} := \frac{1}{\tau} \\
\hfill \\
\Rightarrow n(t) &= n_0 \cdot e^{-\frac{t}{\tau}}
\intertext{Wir rechnen ein Beispiel mit $N_2$ Gas bei $T = \unit[300]{K}$ und $P = \unit[1]{mbar}$, dabei ist $<v> = \unit[426]{m/s}$.}
\tau &= \frac{4 \cdot 1}{426 \cdot \frac{\left( 10^{-3} \right)^2 \pi}{4}} = \unit[1,2 \cdot 10^4]{s}
\intertext{Dieses $\tau$ nutzen wir jetzt für die Bestimmung des $\unit[50]{\%}$ Wertes:}
\half n_0 &= n_0 \cdot e^{- \frac{t}{\tau}} \\
\Leftrightarrow t &= -\tau \ln\left( \half \right) = \unit[8,3 \cdot 10^3]{s}
\end{align*}










\end{document}