\documentclass[11pt]{scrartcl}
\usepackage[T1]{fontenc}
\usepackage[a4paper, left=3cm, right=2cm, top=2cm, bottom=2cm]{geometry}
\usepackage[activate]{pdfcprot}
\usepackage[ngerman]{babel}
\usepackage[parfill]{parskip}
\usepackage[utf8]{inputenc}
\usepackage{kurier}
\usepackage{amsmath}
\usepackage{amssymb}
\usepackage{xcolor}
\usepackage{epstopdf}
\usepackage{txfonts}
\usepackage{fancyhdr}
\usepackage{graphicx}
\usepackage{prettyref}
\usepackage{hyperref}
\usepackage{eurosym}
\usepackage{setspace}
\usepackage{units}
\usepackage{eso-pic,graphicx}
\usepackage{icomma}
\usepackage{pdfpages}

\definecolor{darkblue}{rgb}{0,0,.5}
\hypersetup{pdftex=true, colorlinks=true, breaklinks=false, linkcolor=black, menucolor=black, pagecolor=black, urlcolor=darkblue}



\setlength{\columnsep}{2cm}


\newcommand{\arcsinh}{\mathrm{arcsinh}}
\newcommand{\asinh}{\mathrm{arcsinh}}
\newcommand{\ergebnis}{\textcolor{red}{\mathrm{Ergebnis}}}
\newcommand{\fehlt}{\textcolor{red}{Hier fehlen noch Inhalte.}}
\newcommand{\betanotice}{\textcolor{red}{Diese Aufgaben sind noch nicht in der Übung kontrolliert worden. Es sind lediglich meine Überlegungen und Lösungsansätze zu den Aufgaben. Es können Fehler enthalten sein!!! Das Dokument wird fortwährend aktualisiert und erst wenn das \textcolor{black}{beta} aus dem Dateinamen verschwindet ist es endgültig.}}
\newcommand{\half}{\frac{1}{2}}
\renewcommand{\d}{\, \mathrm d}
\newcommand{\punkte}{\textcolor{white}{xxxxx}}
\newcommand{\p}{\, \partial}
\newcommand{\dd}[1]{\item[#1] \hfill \\}

\renewcommand{\familydefault}{\sfdefault}
\renewcommand\thesection{}
\renewcommand\thesubsection{}
\renewcommand\thesubsubsection{}


\newcommand{\themodul}{Halbleiter und Nanotechnologie}
\newcommand{\thetutor}{Prof. Förster}
\newcommand{\theuebung}{Übung 3}

\pagestyle{fancy}
\fancyhead[L]{\footnotesize{C. Hansen}}
\chead{\thepage}
\rhead{}
\lfoot{}
\cfoot{}
\rfoot{}

\title{\themodul{}, \theuebung{}, \thetutor}


\author{Christoph Hansen \\ {\small \href{mailto:chris@university-material.de}{chris@university-material.de}} }

\date{}


\begin{document}

\maketitle

Dieser Text ist unter dieser \href{http://creativecommons.org/licenses/by-nc-sa/4.0/}{Creative Commons} Lizenz veröffentlicht.

\textcolor{red}{Ich erhebe keinen Anspruch auf Vollständigkeit oder Richtigkeit. Falls ihr Fehler findet oder etwas fehlt, dann meldet euch bitte über den Emailkontakt.}

\tableofcontents


\newpage



\section{Aufgabe 1}


Die Ladungsträgerkonzentration in Kupfer ist $n_{Cu} = \unit[8,43 \cdot 10^{28}]{1/m^3}$. Nun können wir den Druck im Gas bestimmen:

\begin{align*}
P &=n \cdot k \cdot T = 8,43 \cdot 10^{28} \cdot 1,38 \cdot 10^{-23} \cdot 300 = \unit[3,49 \cdot 10^{8}]{Pa} = \unit[3490]{bar}
\end{align*}


\section{Aufgabe 2}

Das geht einfach über die Energiegleichung:

\begin{align*}
E_{Cu} &= \half mv^2 = 7 \cdot 1,6 \cdot 10^{-19} \\
\Leftrightarrow v &= \sqrt{\frac{2 \cdot 7 \cdot 1,6 \cdot 10^{-19}}{9,1 \cdot 10^{-31}}} = \unit[1,6 \cdot 10^6]{m/s} 
\end{align*}


\section{Aufgabe 3}

Für die Zustandsdichte gilt:

\begin{align*}
D(E) &= \frac{\d N}{\d E} = \frac{1}{2 \pi^2} \cdot \left( \frac{2m}{\hbar^2} \right)^{\frac{3}{2}} \cdot \sqrt{E} \cdot V = c_1 \cdot V \cdot \sqrt{E}
\intertext{Dabei ist $\d N$ die Anzahl der Elektronen im Energieintervall. Nun berechnen wir die Teilchendichte $N$:}
N &= V \cdot \int_{0}^{E_F} D(E) \d E = \frac{V}{2 \pi^2} \cdot \left( \frac{2m}{\hbar^2} \right)^{\frac{3}{2}} \cdot \int_{0}^{E_F} \sqrt{E} \d E \\
&= \frac{2}{3} \left[ \frac{V}{2\pi^2} \left( \frac{2m}{\hbar^2} \right)^{\frac{3}{2}} \right] \cdot E_F^{\frac{3}{2}} = \frac{2}{3} \cdot c_1 \cdot E^{\frac{3}{2}} \cdot V
\intertext{Die mittlere Energie ist:}
<E> &= \frac{V}{N} \cdot \int_{0}^{E_F} E \cdot D(E) \d E = \frac{V}{N} \cdot \int_{0}^{E_F} E \cdot c_1 \cdot \sqrt{E} \d E \\
&= \frac{V}{N} \cdot c_1 \cdot \int_{0}^{E_F} E^{\frac{3}{2}} \d E = \frac{V}{N} \cdot c_1 \cdot \frac{2}{5} E^{\frac{5}{2}} \\
&= \frac{V}{\frac{2}{3} \cdot c_1 \cdot E^{\frac{3}{2}} \cdot V} \cdot c_1 \cdot \frac{2}{5} \cdot E^{\frac{5}{2}} = \frac{3}{5} E_F
\end{align*}


\section{Aufgabe 4}

\subsection*{a)}

Wir berechnen zunächst die Masse eines Atoms:

\begin{align*}
m_{Al} &= \frac{M_{Al}}{N_A} = \frac{26,98}{6,022 \cdot 10^{23}} = \unit[4,48 \cdot 10^{-26}]{kg}
\intertext{Nun noc das Volumen:}
V &= \frac{m_{Al}}{\rho} = \frac{4,48 \cdot 10^{-26}}{2,7 \cdot 10^3} = \unit[1,66 \cdot 10^{-29}]{m^3}
\intertext{Die Elektronendichte ist nun:}
n &= \frac{3}{V} = \frac{3}{1,66 \cdot 10^{-29}} = \unit[1,8 \cdot 10^{29}]{1/m^3}
\end{align*}


\subsection*{b)}

Wir nehmen uns die Formel für die Elektronendichte aus der Aufgabe 3 und formen nach der Fermienergie um:

\begin{align*}
n &= \frac{2}{3} \cdot c_1 \cdot E_F^{\frac{3}{2}} \qquad \Leftrightarrow E_F = \left( \frac{3}{2} \cdot \frac{1}{c_1} \cdot n \right)^{\frac{2}{3}}
\intertext{Wir setzen $c_1$ ein:}
E_F &= \left[ \frac{3}{2} \cdot 2 \pi^2 \cdot \left( \frac{\hbar^2}{2m} \right)^{\frac{3}{2}} \cdot n \right]^{\frac{2}{3}} = \left( 3 \pi^2 \right)^{\frac{2}{3}} \cdot \frac{\hbar^2}{2m} \cdot n^{\frac{2}{3}} = \unit[11,6]{eV}
\end{align*}

\subsection*{c)}

\begin{align*}
E_F &= k \cdot T_F \\
\Leftrightarrow T &= \frac{E_F}{k} = \frac{1,86 \cdot 10^{-19}}{1,38 \cdot 10^{-23}} = \unit[135000]{K}
\end{align*}

\section{Aufgabe 5}

Die gesamte innere Energie ist:

\begin{align*}
U &= \frac{3}{5} \cdot N \cdot E_F = N \cdot \frac{3}{5} \cdot \frac{\hbar^2}{2m} \cdot \left( 3 \pi^2 \cdot n \right)^{\frac{2}{3}} \qquad \text{mit} \qquad n = \frac{N}{V} 
\intertext{Dies leiten wir nun ab:}
P &= \frac{\p U}{\p V} = \frac{2}{5} \cdot n \cdot E_F
\intertext{Mit den Werten aus A4 ergibt sich ein Druck:}
P &= \frac{2}{5} \cdot 1,8 \cdot 10^{29} \cdot 11,6 \cdot 1,6 \cdot 10^{-19} = \unit[1,3 \cdot 10^{11}]{Pa} = \unit[1,3 \cdot 10^6]{bar}
\end{align*}


\section{Aufgabe 6}

Wir schreiben $N_C$ und $N_V$ aus:

\begin{align*}
N_V &= 2 \cdot \left( \frac{kT}{2 \pi \hbar} \right)^{3/2} \cdot m*_h^{3/2} \\
N_C &= 2 \cdot \left( \frac{kT}{2 \pi \hbar} \right)^{3/2} \cdot m*_e^{3/2}
\intertext{Das können wir einsetzen:}
\ln\left( \frac{N_V}{N_C} \right) &= \ln\left( \left( \frac{m_h*}{m_e*} \right)^{3/2} \right) = \frac{3}{2} \cdot \ln\left( \frac{m_h*}{m_e*} \right)
\intertext{Die entgültige Formel lautet dann:}
E_f &= \frac{E_C - E_V}{2} + \frac{3kT}{4} \cdot \ln\left( \frac{m_h*}{m_e*} \right)
\end{align*}

Das gilt allerdings nur für intrinsische Halbleiter.


\section{Aufgabe 7}

\subsection*{a)}

Die Bindungsenergie für Elektronen ist:

\begin{align*}
E_B &= - \frac{m_e^*}{\epsilon_r^2} \cdot \unit[13,6]{eV}
\intertext{Der Bindungsradius ist:}
r_B &= \frac{\epsilon_r}{\frac{m^*}{m_e}} \cdot r_{Bohr} \qquad \text{mit} \qquad r_{Bohr} = \unit[5,29 \cdot 10^{-11}]{m}
\end{align*}

Den Rest der Aufgabe fand ich in der Übung sehr schlecht präsentiert und ich bin leider nicht in der Lage es sinnvoll aufzubereiten. Wenn jemand eine schöne Darstellung hat dann soll er sich bitte bei mit melden.




















\end{document}