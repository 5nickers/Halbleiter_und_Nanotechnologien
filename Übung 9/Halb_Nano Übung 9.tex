\input{header.tex}


\begin{document}

\maketitle

Dieser Text ist unter dieser \href{http://creativecommons.org/licenses/by-nc-sa/4.0/}{Creative Commons} Lizenz veröffentlicht.

\textcolor{red}{Ich erhebe keinen Anspruch auf Vollständigkeit oder Richtigkeit. Falls ihr Fehler findet oder etwas fehlt, dann meldet euch bitte über den Emailkontakt.}

\tableofcontents


\newpage



\section{Aufgabe 1}

\subsection*{a)}

\begin{figure}[h]
	\centering
	\includegraphics[scale=0.1]{A1_1.jpg}
\end{figure}

Pinch-Off Spannung heißt, das die Raumladungszone verarmt ist. Diese Spannung können wir so berechnen:

\begin{align*}
V_P &= \frac{a^2 e N_d}{2 \cdot \epsilon_0 \cdot \epsilon_r} = \frac{\left( 400 \cdot 10^{-9} \right)^2 \cdot 1,6 \cdot 10^{-19} \cdot 10^{23}}{2 \cdot 8,85 \cdot 10^{-12} \cdot 12,9} = \unit[11,2]{V}
\end{align*}


\subsection*{b)}

\begin{align*}
I_{Sä} &= I_P \cdot \left[ \frac{3 \cdot \left( V_P - V_g - V_{bi} \right)}{V_P} - \frac{2 \cdot \left( V_P^{3/2} - \left( V_g - V_{bi} \right)^{3/2} \right)}{V_P^{3/2}} \right]
\intertext{Dabei ist der Pinch-Off Strom:}
I_P &= \frac{z \cdot \mu \cdot q^2 \cdot N_d^2 \cdot a^3}{6 \cdot \epsilon_0 \epsilon_r \cdot L} = \frac{30 \cdot 10^{-6} \cdot 0,2 \cdot \left( 1,6 \cdot 10^{-19} \right)^2 \cdot \left( 10^{23} \right)^2 \cdot \left( 400 \cdot 10^{-9} \right)^3}{6 \cdot 8,85 \cdot 10^{-12} \cdot 12,9 \cdot 3 \cdot 10^{-6}} = \unit[47,8]{mA}
\intertext{Die Sättigungsspannungen sind dann:}
I_{V_g = 2} &= 47,8 \cdot \left[ \frac{3 \cdot \left( 11,2 - 2 - 0,6 \right)}{11,2} - \frac{2 \cdot \left( 11,2^{3/2} - \left( 2 - 0,6 \right)^{3/2} \right)}{11,2^{3/2}} \right] = \unit[25,2]{mA} \\
I_{V_g = 4} &= 47,8 \cdot \left[ \frac{3 \cdot \left( 11,2 - 4 - 0,6 \right)}{11,2} - \frac{2 \cdot \left( 11,2^{3/2} - \left( 4 - 0,6 \right)^{3/2} \right)}{11,2^{3/2}} \right] = \unit[14]{mA} \\
I_{V_g = 6} &= 47,8 \cdot \left[ \frac{3 \cdot \left( 11,2 - 6 - 0,6 \right)}{11,2} - \frac{2 \cdot \left( 11,2^{3/2} - \left( 6 - 0,6 \right)^{3/2} \right)}{11,2^{3/2}} \right] = \unit[6]{mA} 
\end{align*}

\subsection*{c)}

Die Drainströme werden bei folgenden Spannungen erreicht:

\begin{align*}
V_{DS} &= V_P - V_{bi} - V_g \\
\hfil \\
V_{DS,2} &= 11,2 - 0,6 - 2 = \unit[8,6]{V} \\
V_{DS,4} &= 11,2 - 0,6 - 4 = \unit[6,6]{V} \\
V_{DS,6} &= 11,2 - 0,6 - 6 = \unit[4,6]{V} \\
\end{align*}


\subsection*{d)}

\begin{figure}[h]
	\centering
	\includegraphics[scale=0.1]{A1_2.jpg}
\end{figure}


\subsection*{e)}

Die Steilheiten können wir wieder mit einer Formel berechnen:

\begin{align*}
-g_{m,2} &= I_P \cdot \left[ \frac{-3}{V_P} + \frac{3 \cdot \sqrt{V_g + V_{bi}}}{V_P^{3/2}} \right] = 47,8 \cdot \left[ \frac{-3}{11,2} + \frac{3 \cdot \sqrt{2 + 0,6}}{11,2^{3/2}} \right] = \unit[6,6]{mA/V} \\
-g_{m,4} &= 47,8 \cdot \left[ \frac{-3}{11,2} + \frac{3 \cdot \sqrt{4 + 0,6}}{11,2^{3/2}} \right] = \unit[4,6]{mA/V} \\
-g_{m,2} &=  47,8 \cdot \left[ \frac{-3}{11,2} + \frac{3 \cdot \sqrt{4 + 0,6}}{11,2^{3/2}} \right] = \unit[2,8]{mA/V}
\end{align*}


\newpage

\section{Aufgabe 2}

\subsection*{a)}

\begin{figure}[h]
	\centering
	\includegraphics[scale=0.13]{A2_1.jpg}
\end{figure}


\subsection*{b)}

\begin{figure}[h]
	\centering
	\includegraphics[scale=0.13]{A2_2.jpg}
\end{figure}


\newpage

\subsection*{c)}

\begin{figure}[h]
	\centering
	\includegraphics[scale=0.13]{A2_3.jpg}
\end{figure}

Wenn wir noch höhere Spannungen anlegen, dann kommt es zur Invresion an der Grenzfläche. Das ist n-Leitung und in \textcolor{red}{rot} gekennzeichnet.



\section{Aufgabe 3}

\subsection*{a)}

\begin{figure}[h]
	\centering
	\includegraphics[scale=0.13]{A3_1.jpg}
	\caption{Arbeitsgrade mit Ausgangsspannung in grün}
\end{figure}


Die Spannung am Lastwiderstand bestimmen wir über das ohmsche Gesetz:

\begin{align*}
U_L &= R_L \cdot I_L = 250 \cdot 25 \cdot 10^{-3} = \unit[6,25]{V}
\intertext{Damit fällt für die Ausgangsspannung noch ab:}
U_{out} &= 20 - 4,25 = \unit[13,75]{V}
\end{align*}



\subsection*{b)}

in grün im ersten Bild.

\subsection*{c)}


\begin{figure}[h]
	\centering
	\includegraphics[scale=0.13]{A3_2.jpg}
\end{figure}


\subsection*{d)}

Nein, da hier die negative Halbwelle größer ist als die positive.

\subsection*{e)}

\begin{align*}
\frac{\Delta U_{out}}{\Delta U_{in}} = \frac{3}{1} = 1,5
\end{align*}


\section{Aufgabe 4}

wurde nicht besprochen.












\end{document}