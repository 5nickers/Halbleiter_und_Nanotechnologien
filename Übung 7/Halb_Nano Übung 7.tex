\documentclass[11pt]{scrartcl}
\usepackage[T1]{fontenc}
\usepackage[a4paper, left=3cm, right=2cm, top=2cm, bottom=2cm]{geometry}
\usepackage[activate]{pdfcprot}
\usepackage[ngerman]{babel}
\usepackage[parfill]{parskip}
\usepackage[utf8]{inputenc}
\usepackage{kurier}
\usepackage{amsmath}
\usepackage{amssymb}
\usepackage{xcolor}
\usepackage{epstopdf}
\usepackage{txfonts}
\usepackage{fancyhdr}
\usepackage{graphicx}
\usepackage{prettyref}
\usepackage{hyperref}
\usepackage{eurosym}
\usepackage{setspace}
\usepackage{units}
\usepackage{eso-pic,graphicx}
\usepackage{icomma}
\usepackage{pdfpages}

\definecolor{darkblue}{rgb}{0,0,.5}
\hypersetup{pdftex=true, colorlinks=true, breaklinks=false, linkcolor=black, menucolor=black, pagecolor=black, urlcolor=darkblue}



\setlength{\columnsep}{2cm}


\newcommand{\arcsinh}{\mathrm{arcsinh}}
\newcommand{\asinh}{\mathrm{arcsinh}}
\newcommand{\ergebnis}{\textcolor{red}{\mathrm{Ergebnis}}}
\newcommand{\fehlt}{\textcolor{red}{Hier fehlen noch Inhalte.}}
\newcommand{\betanotice}{\textcolor{red}{Diese Aufgaben sind noch nicht in der Übung kontrolliert worden. Es sind lediglich meine Überlegungen und Lösungsansätze zu den Aufgaben. Es können Fehler enthalten sein!!! Das Dokument wird fortwährend aktualisiert und erst wenn das \textcolor{black}{beta} aus dem Dateinamen verschwindet ist es endgültig.}}
\newcommand{\half}{\frac{1}{2}}
\renewcommand{\d}{\, \mathrm d}
\newcommand{\punkte}{\textcolor{white}{xxxxx}}
\newcommand{\p}{\, \partial}
\newcommand{\dd}[1]{\item[#1] \hfill \\}

\renewcommand{\familydefault}{\sfdefault}
\renewcommand\thesection{}
\renewcommand\thesubsection{}
\renewcommand\thesubsubsection{}


\newcommand{\themodul}{Halbleiter und Nanotechnologie}
\newcommand{\thetutor}{Prof. Förster}
\newcommand{\theuebung}{Übung 3}

\pagestyle{fancy}
\fancyhead[L]{\footnotesize{C. Hansen}}
\chead{\thepage}
\rhead{}
\lfoot{}
\cfoot{}
\rfoot{}

\title{\themodul{}, \theuebung{}, \thetutor}


\author{Christoph Hansen \\ {\small \href{mailto:chris@university-material.de}{chris@university-material.de}} }

\date{}


\begin{document}

\maketitle

Dieser Text ist unter dieser \href{http://creativecommons.org/licenses/by-nc-sa/4.0/}{Creative Commons} Lizenz veröffentlicht.

\textcolor{red}{Ich erhebe keinen Anspruch auf Vollständigkeit oder Richtigkeit. Falls ihr Fehler findet oder etwas fehlt, dann meldet euch bitte über den Emailkontakt.}

\tableofcontents


\newpage



\section{Aufgabe 1}

\subsection*{a)}

Es gibt zwar 6 Taschen, aber nur zwei unterschiedliche Massen bei den Elektronen. Diese sind $m_t = transversal$ und $m_l = longitudinal$.


\subsection*{b)}

Zunächst setzen wir die gegebenen Größen ein:


\begin{align*}
\omega^2 &= \frac{e^2 B^2}{m_t^2} \cdot \cos^2(\theta) + \frac{e^2 B^2}{m_t m_l} \cdot \sin^2(\theta) := \frac{e^2 B^2}{m*^2} \\
&= e^2 B^2 \cdot \left( \frac{\cos^2(\theta)}{m_t^2} + \frac{\sin^2(\theta)}{m_t m_l} \right) = e^2 B^2 \cdot \frac{}{^m*^2} \\
\Leftrightarrow \frac{1}{m*^2} &= \frac{\cos^2(\theta)}{m_t^2} + \frac{\sin^2(\theta)}{m_t m_l}
\end{align*}


\section{Aufgabe 2}

\begin{align*}
E_F &= E_D - 3kT 
\intertext{Wir rechnen mit $kT = \unit[0,026]{eV}$ bei $RT = \unit[26]{meV}$. Dann gilt für das Niveau $3kT$ unterhalb des Donatorzustandes:}
n_d &= N_D \cdot \frac{1}{1 + \half e^{\frac{E - E_F}{kT}}} = N_D \cdot \frac{1}{1 + \half e^{\frac{E_F + 3kT - E_F}{kt}}} = N_D \cdot \frac{1}{1 + \half e^3} = N_D \cdot 9,056 \cdot 10^{-2} = \unit[9,05 \cdot 10^{15}]{cm^3}
\intertext{Für dn Zustand $3kT$ oberhalb des Donatorzustandes gilt dann:}
n_d &= N_D \cdot \frac{1}{1 + \half e^{\frac{E_F - 3kT + E_F}{kt}}} = \frac{1}{1 + \half e^{-3}} \cdot N_D = \unit[9,76 \cdot 10^{16}]{cm^3}
\end{align*}











 



\end{document}