\input{header.tex}


\begin{document}

\maketitle

Dieser Text ist unter dieser \href{http://creativecommons.org/licenses/by-nc-sa/4.0/}{Creative Commons} Lizenz veröffentlicht.

\textcolor{red}{Ich erhebe keinen Anspruch auf Vollständigkeit oder Richtigkeit. Falls ihr Fehler findet oder etwas fehlt, dann meldet euch bitte über den Emailkontakt.}

\tableofcontents


\newpage



\section{Aufgabe 1}

\subsection*{a)}

Es gibt zwar 6 Taschen, aber nur zwei unterschiedliche Massen bei den Elektronen. Diese sind $m_t = transversal$ und $m_l = longitudinal$.


\subsection*{b)}

Zunächst setzen wir die gegebenen Größen ein:


\begin{align*}
\omega^2 &= \frac{e^2 B^2}{m_t^2} \cdot \cos^2(\theta) + \frac{e^2 B^2}{m_t m_l} \cdot \sin^2(\theta) := \frac{e^2 B^2}{m*^2} \\
&= e^2 B^2 \cdot \left( \frac{\cos^2(\theta)}{m_t^2} + \frac{\sin^2(\theta)}{m_t m_l} \right) = e^2 B^2 \cdot \frac{}{^m*^2} \\
\Leftrightarrow \frac{1}{m*^2} &= \frac{\cos^2(\theta)}{m_t^2} + \frac{\sin^2(\theta)}{m_t m_l}
\end{align*}


\section{Aufgabe 2}

\subsection*{a)}

\begin{align*}
E_F &= E_D - 3kT 
\intertext{Wir rechnen mit $kT = \unit[0,026]{eV}$ bei $RT = \unit[26]{meV}$. Dann gilt für das Niveau $3kT$ unterhalb des Donatorzustandes:}
n_d &= N_d \cdot \frac{1}{1 + \half e^{\frac{E - E_F}{kT}}} = N_d \cdot \frac{1}{1 + \half e^{\frac{E_F + 3kT - E_F}{kt}}} = N_d \cdot \frac{1}{1 + \half e^3} = N_d \cdot 9,056 \cdot 10^{-2} = \unit[9,05 \cdot 10^{15}]{cm^3}
\intertext{Für dn Zustand $3kT$ oberhalb des Donatorzustandes gilt dann:}
n_d &= N_d \cdot \frac{1}{1 + \half e^{\frac{E_F - 3kT + E_F}{kt}}} = \frac{1}{1 + \half e^{-3}} \cdot N_d = \unit[9,76 \cdot 10^{16}]{cm^3}
\intertext{Im Falle von A1 haben wir nun eine Dichte (freie Elektronen im Leitungsband) von:}
n &= N_d - n_d = 1,01 \cdot 10^{17} - 9,05 \cdot 10^{15} = \unit[9,09 \cdot 10^{16}]{cm^{-3}}
\intertext{Bei A2 gilt:}
n &= N_d - n_d =  10^{17} - 9,76 \cdot 10^{16} = \unit[2,4 \cdot 10^{15}]{cm^{-3}}
\end{align*}


\newpage

\subsection*{b)}

Bei sehr kleinen Temperaturen ist das Ferminiveau nur minimal über dem dem Donatorniveau. Die Elektronenkonzentration ändert sich nun ganz sprunghaft von $\unit[9,09 \cdot 10^{16}]{cm^{-3}}$ auf $\unit[2,4 \cdot 10^{15}]{cm^{-3}}$. In ein Bild gefasst sieht das ca so aus:

\begin{figure}[h]
	\centering
	\includegraphics[scale=0.1]{A2_1.jpg}
\end{figure}


\subsection*{c)}

Hier müssen wir einfach in eine Formel einsetzen:

\begin{align*}
N_c &= 2 \cdot \left( \frac{2 \pi \cdot m^* \cdot kT}{h^2} \right)^\frac{3}{2} = 2 \cdot \left( \frac{2 \pi \cdot 0,89 \cdot 9,1 \cdot 10^{-31} \cdot 1,38 \cdot 10^{-23} \cdot 300}{\left( 6,62 \cdot 10^{-34} \right)^2} \right)^\frac{3}{2} = \unit[2,43 \cdot 10^{25}]{m^{-3}}
\end{align*}


\subsection*{d)}

Zunächst ein Bild zu besseren Vorstellung:

\begin{figure}[h]
	\centering
	\includegraphics[scale=0.1]{A2_2.jpg}
\end{figure}


Nun setzen wir ein:

\begin{align*}
n &= N_c \cdot e^{- \frac{E_c - E_F}{kT}} = 2,43 \cdot 10^{25} \cdot e^{- \frac{123}{k \cdot 300}} = \unit[2,14 \cdot 10^{23}]{m^{-3}}
\end{align*}


\subsection*{e)}

In der Übung konnte die Lösung zu diesem Teil leider nicht sinnvoll geklärt werden.


\section{Aufgabe 3}

Diese Aufgabe ist im Skript gelöst. Zu finden ist die Lösung im Kapitel 2.5 Heterostrukturen und Schottky-Kontakt. Das ist Seite 157 in dem Sktipt auf meiner Website.


\section{Aufgabe 4}


Die Dicke der Schicht können wir so berechnen:

\begin{align*}
w &= \sqrt{\frac{2 \cdot \epsilon_r \cdot \epsilon_0 \cdot \left( V_{bi} + V_R \right)}{e \cdot N_d}} \qquad \text{Dabei ist  $V_R$ die Gatespannung mit - am Gate}
\intertext{Die Kapazität können wir so bestimmen:}
C &= \frac{\epsilon_0 \cdot \epsilon_r \cdot A}{d} = \frac{\epsilon_0 \cdot \epsilon_r \cdot A}{w} \\
\Rightarrow C' &= \frac{C}{A} = \frac{\epsilon_0 \epsilon_r}{w} = \sqrt{\frac{e \cdot N_d \cdot \epsilon_0 \epsilon_r}{\left( V_{bi} + V_R \right)^2}}
\intertext{Zudem gilt:}
\frac{1}{C'^2} \sim V_{bi} + V_R
\end{align*}

Damit können wir jetzt eine Tabelle mit den Werte aufstellen:

\hfill \\

\begin{center}
	\begin{tabular}{c|c|c|c}
		$n/cm^{-3}$ & $w/ nm$ & $V_R$ & $C'/ mF/m^2$ \\ 
		\hline  
	$10^{16}$	& 290 & 0 & 0,39 \\ 
	$10^{16}$	& 890 & 5 & 0,13 \\ 
	$10^{16}$	& $1,2 \cdot 10^{-3}$ & 10 & $93 \cdot 10^{-3}$ \\ 
	$10^{18}$	& 29 & 0 & 3,9 \\ 
	$10^{18}$	& 89 & 5 & 1,3 \\ 
	$10^{18}$	& $0,12 \cdot 10^{-3}$ & 10 & $928 \cdot 10^{-3}$ \\ 	
	\end{tabular} 
\end{center}

\newpage

Als Graphik sieht das dann so aus:

\begin{figure}[h]
	\centering
	\includegraphics[scale=0.15]{A4_1.jpg}
\end{figure}









 



\end{document}