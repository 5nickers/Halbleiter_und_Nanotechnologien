\documentclass[11pt]{scrartcl}
\usepackage[T1]{fontenc}
\usepackage[a4paper, left=3cm, right=2cm, top=2cm, bottom=2cm]{geometry}
\usepackage[activate]{pdfcprot}
\usepackage[ngerman]{babel}
\usepackage[parfill]{parskip}
\usepackage[utf8]{inputenc}
\usepackage{kurier}
\usepackage{amsmath}
\usepackage{amssymb}
\usepackage{xcolor}
\usepackage{epstopdf}
\usepackage{txfonts}
\usepackage{fancyhdr}
\usepackage{graphicx}
\usepackage{prettyref}
\usepackage{hyperref}
\usepackage{eurosym}
\usepackage{setspace}
\usepackage{units}
\usepackage{eso-pic,graphicx}
\usepackage{icomma}
\usepackage{pdfpages}

\definecolor{darkblue}{rgb}{0,0,.5}
\hypersetup{pdftex=true, colorlinks=true, breaklinks=false, linkcolor=black, menucolor=black, pagecolor=black, urlcolor=darkblue}



\setlength{\columnsep}{2cm}


\newcommand{\arcsinh}{\mathrm{arcsinh}}
\newcommand{\asinh}{\mathrm{arcsinh}}
\newcommand{\ergebnis}{\textcolor{red}{\mathrm{Ergebnis}}}
\newcommand{\fehlt}{\textcolor{red}{Hier fehlen noch Inhalte.}}
\newcommand{\betanotice}{\textcolor{red}{Diese Aufgaben sind noch nicht in der Übung kontrolliert worden. Es sind lediglich meine Überlegungen und Lösungsansätze zu den Aufgaben. Es können Fehler enthalten sein!!! Das Dokument wird fortwährend aktualisiert und erst wenn das \textcolor{black}{beta} aus dem Dateinamen verschwindet ist es endgültig.}}
\newcommand{\half}{\frac{1}{2}}
\renewcommand{\d}{\, \mathrm d}
\newcommand{\punkte}{\textcolor{white}{xxxxx}}
\newcommand{\p}{\, \partial}
\newcommand{\dd}[1]{\item[#1] \hfill \\}

\renewcommand{\familydefault}{\sfdefault}
\renewcommand\thesection{}
\renewcommand\thesubsection{}
\renewcommand\thesubsubsection{}


\newcommand{\themodul}{Halbleiter und Nanotechnologie}
\newcommand{\thetutor}{Prof. Förster}
\newcommand{\theuebung}{Übung 3}

\pagestyle{fancy}
\fancyhead[L]{\footnotesize{C. Hansen}}
\chead{\thepage}
\rhead{}
\lfoot{}
\cfoot{}
\rfoot{}

\title{\themodul{}, \theuebung{}, \thetutor}


\author{Christoph Hansen \\ {\small \href{mailto:chris@university-material.de}{chris@university-material.de}} }

\date{}


\begin{document}

\maketitle

Dieser Text ist unter dieser \href{http://creativecommons.org/licenses/by-nc-sa/4.0/}{Creative Commons} Lizenz veröffentlicht.

\textcolor{red}{Ich erhebe keinen Anspruch auf Vollständigkeit oder Richtigkeit. Falls ihr Fehler findet oder etwas fehlt, dann meldet euch bitte über den Emailkontakt.}

\tableofcontents


\newpage



\section{Aufgabe 1}

\subsection*{a)}

Die Leckrate hängt vom Dampfdruck des Öl ab und dessen Volumenstrom:

\begin{align*}
L &= P_{Öl} \cdot q_V 
\intertext{Der Volumenstrom ist dabei:}
q_V &= \frac{<v> \cdot A}{4} = \sqrt{\frac{RT}{2M \pi}} \cdot A = \sqrt{\frac{8,31 \cdot 300}{2 \cdot 26 \cdot 10^{-3} \cdot \pi}} \cdot 10^{-4} = \unit[1,23 \cdot 10^{-2}]{m^3/s}
\intertext{Die Leckrate ist dann:}
L &= 10^{-6} \cdot 10^{-3} \cdot 10^5 \cdot 1,23 \cdot 10^{-2} = \unit[1,23 \cdot 10^{-6}]{Pa \ m^2/s}
\end{align*}


\subsection*{b)}

Der Enddruck setzt sich aus der Leckrate und der Saugrate zusammen:

\begin{align*}
L &= P_E \cdot S \\
\Leftrightarrow P_E &= \frac{L}{S} = \frac{1,23 \cdot 10^{-6}}{0,1} = \unit[1,23 \cdot 10^{-5}]{Pa}
\end{align*}


\subsection*{c)}

Wir berechnen zunächst den Teilchenstrom, mit dem wir dann den Massestrom berechnen:

\begin{align*}
q_N &= \frac{ n \cdot <v> \cdot A}{4} = n \sqrt{\frac{RT}{2M \pi}} \cdot A = \frac{P}{kT}\sqrt{\frac{RT}{2M \pi}} \cdot A = P \cdot \sqrt{\frac{R}{2 k M \pi T}} \cdot A \\
&= 10^{-6} \cdot 100 \cdot \sqrt{\frac{8,31}{2 \cdot 1,38 \cdot 10^{-23} \cdot 26 \cdot 10^{-3} \cdot \pi \cdot 300}} \cdot 10^{-4} = \unit[2,98 \cdot 10^{14}]{1/s}
\intertext{Der Massenstrom ist dann:}
q_m &= m \cdot q_V = \frac{26 \cdot 10^{-3}}{6,022 \cdot 10^{23}} \cdot 2,98 \cdot 10^{14} = \unit[1,29 \cdot 10^{-11}]{kg/s}
\intertext{Nun müssen wir die Masse des Tröpfchens bestimmen. Dazu bestimmen wir zuerst den Radius:}
10^{-4} &= \frac{4 \pi r^2}{2} \\
r &= \sqrt{\frac{2 \cdot 10^{-4}}{4 \pi}} = \unit[4 \cdot 10^{-3}]{m} \\
m &= \frac{\rho \cdot V}{2} = \frac{800 \cdot \frac{4}{3} \cdot \pi \cdot \left( 4 \cdot 10^{-3} \right)^3}{2} = \unit[10^{-4}]{kg}
\intertext{Die Zeit, die wir brauchen um das Tröpfchen abzusaugen wäre dann:}
t &= \frac{10^{-4}}{1,29 \cdot 10^{-11}} = \unit[8,16 \cdot 10^6]{s} \approx 96
\end{align*}

\subsection*{d)}

Für die Leckrate macht es keinen Unterschied, ob etwas von draußen reinkommt oder von drinnen leckt. Zudem ist es schwer nachzuweisen, ob es ein Leck draußen oder drinnen ist.


\section{Aufgabe 2}

\subsection*{a)}

Den Bedeckungsgrad können wir mittels der Langmuir Isotherme bestimmen:

\begin{align*}
\theta &= \frac{C_L \cdot P}{C_L \cdot P + 1} 
\intertext{Dazu berechnen wir $C_L$:}
C_L &= \frac{\tau_0 \cdot e^{E_{Des}/RT} \cdot H_0 \cdot N_A}{n_{mono} \cdot \sqrt{2 \pi RT \cdot M}} = \frac{10^{-13} \cdot e^{\frac{170000}{8,31 \cdot 723} \cdot 0,5 \cdot 6,022 \cdot 10^{23}}}{10^{19} \cdot \sqrt{2 \pi 8,31 \cdot 723 \cdot 28 \cdot 10^{-3}}} = \unit[180]{1/Pa}
\intertext{Damit ergibt sich der Bedeckungsgrad:}
\theta &= \frac{180 \cdot 10^{-6} \cdot 100}{180 \cdot 10^{-6} \cdot 100 + 1} = 0,0177 \approx \unit[1,8]{\%}
\end{align*}

\newpage

\subsection*{b)}

Hier sollen wir den Druck ausrechnen bei dem eine vorgegebene Bedeckung erreicht wird. Dazu nutzen wir die Formel aus Teil a):

\begin{align*}
\theta &= \frac{C_L \cdot P}{1 + C_L \cdot P} := 0,9 \\
\Leftrightarrow C_L P &= \left( 1 + C_L P \right) \cdot 0,9 \\
\Leftrightarrow 0,9 &= C_L P \cdot \left( 1 - 0,9 \right) \\
\Leftrightarrow P &= \frac{0,9}{C_L \cdot 0,1} = \frac{0,9}{180 \cdot 0,1} = \unit[5 \cdot 10^{-2}]{Pa} = \unit[5 \cdot 10^{-4}]{mbar}
\end{align*}


\section{Aufgabe 3}

\subsection*{c)}

Im Prinzip ist das nur einsetzen und dann lösen der DGL:

\begin{align*}
j_{ads} &= j_N \cdot H_0 \cdot \left( 1 - \theta \right) = \frac{\p \tilde{n}}{\p t} \\
\Leftrightarrow \dot{\tilde{n}} &= j_N \cdot H_0 \cdot \left( 1 - \theta \right) \\
\Leftrightarrow 0 &= j_N \cdot H_0 - j_N \cdot \theta \cdot H_0 - \dot{\tilde{n}} \\
\intertext{Wir setzen für $\theta$ ein:}
\Leftrightarrow 0 &= j_N \cdot H_0 - \frac{j_N \cdot \tilde{n} \cdot H_0}{n_0} - \dot{\tilde{n}} 
\intertext{$n_0$ entspricht dabei einer Monolage Stickstoff.}
\Leftrightarrow 0 &= \dot{\tilde{n}} + \frac{j_N \cdot \tilde{n} \cdot H_0}{n_0} - j_N \cdot H_0
\intertext{Wir nutzen die vorgegebene Lösung für die DGL und bestimmen $c_1$ und $c_2$:}
c_1 &= \frac{j_N \cdot H_0}{n_0} \qquad c_2 = - j_n \cdot H_0 \\
\intertext{Wir setzen für den rechten Teil der Lösung ein:}
- \frac{c_2}{c_1} &= \frac{- j_n \cdot H_0 \cdot n_0}{j_N \cdot H_0} = n_0 \\
\intertext{Damit ist die allgemeine Lösung:}
\Rightarrow \tilde{n} &= A e^{-c_1 t} + n_0
\intertext{Nun bestimmen wir $A$ über den Anfangswert:}
\tilde{n}(0) &= 0 \qquad \rightarrow A + n_0 = 0 \qquad \rightarrow A = -n_0 \\
\Rightarrow \tilde{n} &= n_0 \left(1 - e^{-c_1 t} \right) 
\intertext{Nun können wir $c_1$ genau bestimmen:}
c_1 &= \frac{j_N \cdot H_0}{n_0} = \frac{n \cdot <v> \cdot H_0}{4 \cdot n_0} = \frac{P}{kT} \cdot \frac{<v>}{4} \cdot \frac{H_0}{n_0} \\
&= \frac{P \sqrt{8 kT}}{4kT \sqrt{\pi M}} \cdot \frac{H_0}{n_0} = \frac{P \sqrt{8}}{4 \sqrt{RT \pi M}} \cdot \frac{H_0}{n_0}
\intertext{Wir nutzen $k\cdot m = \frac{RM}{N_A}$}
&= \frac{P \sqrt{8} N_A H_0}{4 \sqrt{RM \pi T} n_0} = \frac{10^{-4} \cdot \sqrt{8} \cdot 6,022 \cdot 10^{23} \cdot 0,5}{4 \cdot \sqrt{8,31 \cdot 28 \cdot 10^{-3} \cdot \pi \cdot 573 \cdot 10^{19}}} = 1,04 \cdot 10^{-1}
\intertext{Damit is unsere Lösung:}
\tilde{n} &= n_0 \left(1 - e^{-0,104 \cdot t} \right) 
\intertext{Um die Zeit für eine $\unit[50]{\%}$ zu berechnen gehen wir wie folgt vor:}
0,5 &= 1 - e^{-0,104 \cdot t} \\
\Leftrightarrow 0,5 &= e^{-0,104 \cdot t} \\
\Leftrightarrow -0,104 t &= \ln(0,5) \\
\Leftrightarrow t &= \frac{\ln(0,5)}{-0,104} = \unit[6,66]{s}
\end{align*}


\subsection*{d)}

Wir betrachten nur die Desorbtion:

\begin{align*}
\tilde{n} &= \tilde{n_0} \cdot e^{-\frac{t}{\tau}} \\
\tau &= \tau_0 \cdot e^{\frac{E_{des}}{RT}} = 10^{-13} \cdot e^{\frac{170000}{8,31 \cdot 573}} = \unit[334]{s}
\Rightarrow \tilde{n} &= \tilde{n_0} \cdot e^{-\frac{t}{334}}
\intertext{Nun berechnen wir wie lange wir warten müssen bis die Bedeckung nur noch $\unit[10]{\%}$ beträgt:}
0,1 &= e^{- \frac{t}{334}} \\
\Leftrightarrow \ln(0,1) &= - \frac{t}{334} \\
\Leftrightarrow t &= \ln(0,1) \cdot 334 = \unit[769]{s}
\end{align*}


\section{Aufgabe 4}


\subsection*{1)}

Bereich Atmosphäre: a,c,f \\

Bereich $10^{-3} - 10^{-8}$: b,d,e (Ionenzerstäuber),h \\

Bereich UHV: e(Ionengetter),g,h,d


\subsection*{2)}

Beim Abpumpen von kondensierbaren Gasen.

\subsection*{3)}

Alle Pumpen in den Bereichen $10^{-3}$ bis UHV brauchen eine Vorpumpe.

\subsection*{4)}

Die Kombination a), da diese Ölfrei ist und daher kein Öl die Kammer verschmutzt.

\subsection*{5)}

\begin{align*}
\unit[10^{-8}]{mbar} &\Rightarrow \unit[100]{s} \\
\unit[10^{-9}]{mbar} &\Rightarrow \unit[1000]{s} \\
\unit[10^{-10}]{mbar} &\Rightarrow \unit[1000]{s} \\
\end{align*}













\end{document}