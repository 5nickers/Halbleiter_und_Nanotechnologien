\input{header.tex}


\begin{document}

\maketitle

Dieser Text ist unter dieser \href{http://creativecommons.org/licenses/by-nc-sa/4.0/}{Creative Commons} Lizenz veröffentlicht.

\textcolor{red}{Ich erhebe keinen Anspruch auf Vollständigkeit oder Richtigkeit. Falls ihr Fehler findet oder etwas fehlt, dann meldet euch bitte über den Emailkontakt.}

\tableofcontents


\newpage



\section{Aufgabe 1}

\subsection*{a)}

\begin{align*}
P_1 V &= \nu R T_1 \\
P_2 V &= \nu R T_2 \\
\intertext{Wir setzen ein:}
T_2 &= \frac{P_2 V}{\nu R} = \frac{P_2 T}{P_1} = \frac{P_2}{P_1} \cdot T_1
\intertext{Wenn nun bei $T_1$ der Druck bekannt ist, dann kann man $P_2$ messen und die Temperatur $T_2$ berechnen. $P_2$ wird gemessen über:}
P_2 &= P_1 + \rho_{Hg} \cdot g \cdot \Delta h
\end{align*}


\subsection*{b)}

Wir setzen die Werte in SI Einheiten ein:

\begin{align*}
P_2 &= 10^5 + 13645 \cdot 9,81 \cdot 0,08 = \unit[1,107 \cdot 10^5]{Pa}
\intertext{Die Temperatur ist dann:}
T_2 &= \frac{1,107 \cdot 10^5}{10^5} \cdot 300 = \unit[332]{K}
\end{align*}


\section{Aufgabe 2}

Wir stellen zunächst drei Gleichungen auf, die wir dann ineinander einsetzen können, daraus erhalten wir dann die gewünschte Formel:

\begin{align*}
&\text{I)}  & P_T V &= \nu R T = \nu R \left( 273,15 \cdot \theta \right) \\
&\text{II)}  & P_{100} V &= \nu R T_{100} = \nu R \left( 273,15 + 100 \right) \\
&\text{III)}  & P_0 V &= \nu R T_0 = \nu R \left( 273,15 \right)
\end{align*}
Nun ziehen wir III von II ab:
\begin{align*}
\left( P_{100} - P_0 \right) V &= \nu R \cdot 100 \\
\Leftrightarrow \nu R &= \frac{P_{100} - P_0}{100} \cdot V
\intertext{Wir ziehen nun I von II ab:}
\nu R \theta &= \left( P_T - P_0 \right) \cdot V \\
\Leftrightarrow \frac{P_{100} - P_0}{100} \cdot V \theta &= \left( P_T - P_0 \right) \cdot V \\
\Leftrightarrow \theta &= \frac{P_T - P_0}{P_{100} - P_0} \cdot 100
\end{align*}


\section{Aufgabe 3}

Wir müssen in beiden Aufgabenteilen einfach nur in die ideale Gasgleichung einsetzen.

\subsection*{a)}

\begin{align*}
PV = \nu RT \Leftrightarrow V &= \frac{RT}{P} = \frac{8,3145 \cdot 273,15}{1,013 \cdot 10^5} = \unit[22,4]{l}  
\end{align*}


\subsection*{b)}

\begin{align*}
PV = \nu RT \Leftrightarrow V &= \frac{RT}{P} = \frac{8,3145 \cdot 298,15}{1,013 \cdot 10^5} = \unit[24,79]{l}  
\end{align*}


\section{Aufgabe 4}

Wir nehmen an, das Raumtemperatur $\unit[300]{K}$ und $\unit[1]{atm} = \unit[9,811 \cdot 10^4]{Pa}$:

\subsection*{a)}

Wir berechnen zunächst die molare Masse von $CO_2$:

\begin{align*}
M(CO_2) &= \unit[12]{g} + 2 \cdot \unit[16]{g} = \unit[44]{g}
\intertext{Damit erhalten wir unser $\nu$:}
\nu &= \frac{100}{44} = \unit[2,27]{mol/l}
\intertext{Nun können wir wieder in die ideale Gasgleichung einsetzen:}
V = \frac{\nu RT}{P} &= \frac{2,27 \cdot 8,134 \cdot 300}{9,811 \cdot 10^4} = \unit[57,8]{l}
\end{align*}


\subsection*{b)}

\begin{align*}
P &= \frac{\nu RT}{V} &= \frac{2,27 \cdot 8,134 \cdot 300}{80 \cdot 10^{-3}} = \unit[7,071 \cdot 10^4]{Pa}
\end{align*}


\section{Aufgabe 5}

\subsection*{a)}

Wir lesen die Drücke aus dem Diagramm ab:

\begin{align*}
P_{294} &= \unit[64,7]{bar} \\
P_{284} &= \unit[55,9]{bar}
\end{align*}


\subsection*{b)}


Wir lesen zwei markante Punkte ab und bauen daraus den Graphen:

\begin{align*}
P_{304} &= \unit[73,4]{bar} \\
P_{274} &= \unit[48,2]{bar}
\end{align*}

Der Graph sollte in etwa linear sein mit der Temperatur auf der x-Achse und einer Steigung von $\frac{\Delta P}{\Delta T} = \unit[0,786]{bar/K}$



\subsection*{c)}

Wir müssen uns zunächst klarmachen was $\Delta V$ ist. $\Delta V$ ist die Volumendifferenz zwischen dem gasigen und dem flüssigen Zustand, also gilt $\Delta V = V_g - V_f$. Nun lösen wir nach der Entalpie auf:

\begin{align*}
\Delta H &= \frac{\p P}{\p T} \cdot \Delta V \cdot T
\intertext{Nun bestimmen wir die $\Delta V$ bei den angegebenen Temperaturen:}
\Delta V(\unit[274]{K}) &= \unit[2,22 \cdot 10^{-4}]{m^3/mol} \\
\Delta V(\unit[284]{K}) &= \unit[1,601 \cdot 10^{-4}]{m^3/mol} \\
\Delta V(\unit[294]{K}) &= \unit[1,1 \cdot 10^{-4}]{m^3/mol} 
\intertext{Nun berechnen wir die Enthalpie:}
\Delta H_{\unit[274]{K}} &= 0,786 \cdot 10^5 \cdot 2,22 \cdot 10^{-4} \cdot 274 = \unit[4,78 \cdot 10^3]{J/mol} \\
\Delta H_{\unit[284]{K}} &= 0,786 \cdot 10^5 \cdot 2,22 \cdot 10^{-4} \cdot 284 = \unit[3,57 \cdot 10^3]{J/mol} \\
\Delta H_{\unit[294]{K}} &= 0,786 \cdot 10^5 \cdot 2,22 \cdot 10^{-4} \cdot 294 = \unit[2,31 \cdot 10^3]{J/mol} \\
\end{align*}


\subsection*{d)}

Mit der Annahme $V_g >> V_f$ vereinfacht sich die Gleichung aus dem vorigen Aufgabenteil zu:

\begin{align*}
\frac{\p P}{\p T} &= \frac{\Delta H}{T V_g} 
\intertext{Nutze $PV = RT$:}
\Leftrightarrow \frac{\p P}{\p T} &= \frac{\Delta H \cdot P}{T^2 R}
\intertext{Teile die Integration auf zwei Seiten auf:}
\frac{\p P}{P} &= \frac{\Delta H \p T}{T^2 R} 
\intertext{Wir führen folgende Integration aus $\int_{P_1}^{P_2}$ und $\int_{T_1}^{T_2}$ und erhalten:}
\ln(P_2) - \ln(P_1) &= - \frac{\Delta H}{R} \left( \frac{1}{T_2} - \frac{1}{T_1} \right) \\
\Leftrightarrow \frac{\p \ln(P)}{\p \frac{1}{T}} &= - \frac{\Delta H}{R}
\end{align*}

\begin{figure}[h]
	\centering
	\includegraphics[scale=0.15]{A5_1.jpg}
	\caption{$\frac{\Delta \ln(P)}{\Delta \frac{1}{T}} = - \frac{\Delta H}{R}$}
\end{figure}


\newpage

\subsection*{e)}

Wir bauen uns zuerst eine Tabelle mit ein paar Eckdaten, die wir schon kennen:

\hfil \\

\begin{center}
	\begin{tabular}{|c|c|c|c|}
		$T$ & $P_D$ & $\ln(P_D)$ & $\frac{1}{T}$ \\ 
		\hline
		$274$ & $48,2 \cdot 10^5$ & $15,39$ & $3,65 \cdot 10^{-3}$ \\ 
		$284$ & $55,9 \cdot 10^5$ & $15,54$ & $3,52 \cdot 10^{-3}$ \\ 
		$294$ & $64,7 \cdot 10^5$ & $15,68$ & $3,4 \cdot 10^{-3}$ \\ 
	\end{tabular} 
\end{center}

\hfil \\

Man erhält dann diesen Graphen:

\begin{figure}[h]
	\centering
	\includegraphics[scale=0.15]{A5_2.jpg}
	\caption{$\frac{\Delta \ln(P)}{\Delta \frac{1}{T}} = -1,16 \cdot 10^{-3}$}
\end{figure}

\begin{align*}
\Delta H = R \cdot 1,16 \cdot 10^{3} = \unit[9,64]{kJ/mol} = \unit[219]{kJ/kg}
\end{align*}


\newpage


\subsection*{f)}

\begin{figure}[h]
	\centering
	\includegraphics[scale=0.15]{A5_3.jpg}
	\caption{}
\end{figure}



\section{Aufgabe 6}


Wir stellen die Gleichung zunächst nach dem Druck um:

\begin{align*}
P &= \frac{R T}{V - b} - \frac{a}{V^2}
\intertext{Weil bei kitischem Volumen ein Sattelpunkt vorliegt, müssen die ersten beiden Ableitungen nach dem Volumen Null sein:}
\frac{\p P}{\p V} &= 0 \qquad \frac{\p^2 P}{\p^2 V} = 0 \\
\frac{\p P}{\p V} &= - \frac{R T}{ (V - b)^2 } + \frac{2 a}{V^3} \\
\Leftrightarrow R T \cdot V^2 &= 2 a (V - b)^2 \\
\hfil \\
\frac{\p^2 P}{\p^2 V} &=  \frac{2 R T}{ (V - b)^3 } - \frac{6 a}{V^4} \\
\Leftrightarrow R T \cdot V^4 &= 3 a (V - b)^3 \\
R T V^3 \cdot V &= 3 a (V - b)^3 
\intertext{Wir setzen nun die erste Ableitung in die zweite ein:}
\Leftrightarrow 2a (V - b)^2 \cdot V &= 3a (V - b)^3 \\
\Leftrightarrow 2 V &= 3 (V - b) \\
\Leftrightarrow V &= 3b
\end{align*}













\end{document}