\input{header.tex}


\begin{document}

\maketitle

Dieser Text ist unter dieser \href{http://creativecommons.org/licenses/by-nc-sa/4.0/}{Creative Commons} Lizenz veröffentlicht.

\textcolor{red}{Ich erhebe keinen Anspruch auf Vollständigkeit oder Richtigkeit. Falls ihr Fehler findet oder etwas fehlt, dann meldet euch bitte über den Emailkontakt.}

\tableofcontents


\newpage



\section{Aufgabe 1}

\subsection*{a)}

\begin{align*}
P_1 V &= \nu R T_1 \\
P_2 V &= \nu R T_2 \\
\intertext{Wir setzen ein:}
T_2 &= \frac{P_2 V}{\nu R} = \frac{P_2 T}{P_1} = \frac{P_2}{P_1} \cdot T_1
\intertext{Wenn nun bei $T_1$ der Druck bekannt ist, dann kann man $P_2$ messen und die Temperatur $T_2$ berechnen. $P_2$ wird gemessen über:}
P_2 &= P_1 + \rho_{Hg} \cdot g \cdot \Delta h
\end{align*}


\subsection*{b)}

Wir setzen die Werte in SI Einheiten ein:

\begin{align*}
P_2 &= 10^5 + 13645 \cdot 9,81 \cdot 0,08 = \unit[1,107 \cdot 10^5]{Pa}
\intertext{Die Temperatur ist dann:}
T_2 &= \frac{1,107 \cdot 10^5}{10^5} \cdot 300 = \unit[332]{K}
\end{align*}


\end{document}