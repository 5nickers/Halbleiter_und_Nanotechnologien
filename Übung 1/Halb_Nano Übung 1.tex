\input{header.tex}


\begin{document}

\maketitle

Dieser Text ist unter dieser \href{http://creativecommons.org/licenses/by-nc-sa/4.0/}{Creative Commons} Lizenz veröffentlicht.

\textcolor{red}{Ich erhebe keinen Anspruch auf Vollständigkeit oder Richtigkeit. Falls ihr Fehler findet oder etwas fehlt, dann meldet euch bitte über den Emailkontakt.}

\tableofcontents


\newpage



\section{Aufgabe 1}

\subsection*{a)}

\begin{align*}
P_1 V &= \nu R T_1 \\
P_2 V &= \nu R T_2 \\
\intertext{Wir setzen ein:}
T_2 &= \frac{P_2 V}{\nu R} = \frac{P_2 T}{P_1} = \frac{P_2}{P_1} \cdot T_1
\intertext{Wenn nun bei $T_1$ der Druck bekannt ist, dann kann man $P_2$ messen und die Temperatur $T_2$ berechnen. $P_2$ wird gemessen über:}
P_2 &= P_1 + \rho_{Hg} \cdot g \cdot \Delta h
\end{align*}


\subsection*{b)}

Wir setzen die Werte in SI Einheiten ein:

\begin{align*}
P_2 &= 10^5 + 13645 \cdot 9,81 \cdot 0,08 = \unit[1,107 \cdot 10^5]{Pa}
\intertext{Die Temperatur ist dann:}
T_2 &= \frac{1,107 \cdot 10^5}{10^5} \cdot 300 = \unit[332]{K}
\end{align*}


\section{Aufgabe 2}

Wir stellen zunächst drei Gleichungen auf, die wir dann ineinander einsetzen können, daraus erhalten wir dann die gewünschte Formel:

\begin{align*}
&\text{I)}  & P_T V &= \nu R T = \nu R \left( 273,15 \cdot \theta \right) \\
&\text{II)}  & P_{100} V &= \nu R T_{100} = \nu R \left( 273,15 + 100 \right) \\
&\text{III)}  & P_0 V &= \nu R T_0 = \nu R \left( 273,15 \right)
\end{align*}
Nun ziehen wir III von II ab:
\begin{align*}
\left( P_{100} - P_0 \right) V &= \nu R \cdot 100 \\
\Leftrightarrow \nu R &= \frac{P_{100} - P_0}{100} \cdot V
\intertext{Wir ziehen nun I von II ab:}
\nu R \theta &= \left( P_T - P_0 \right) \cdot V \\
\Leftrightarrow \frac{P_{100} - P_0}{100} \cdot V \theta &= \left( P_T - P_0 \right) \cdot V \\
\Leftrightarrow \theta &= \frac{P_T - P_0}{P_{100} - P_0} \cdot 100
\end{align*}


\section{Aufgabe 3}

Wir müssen in beiden Aufgabenteilen einfach nur in die ideale Gasgleichung einsetzen.

\subsection*{a)}

\begin{align*}
PV = \nu RT \Leftrightarrow V &= \frac{RT}{P} = \frac{8,3145 \cdot 273,15}{1,013 \cdot 10^5} = \unit[22,4]{l}  
\end{align*}


\subsection*{b)}

\begin{align*}
PV = \nu RT \Leftrightarrow V &= \frac{RT}{P} = \frac{8,3145 \cdot 298,15}{1,013 \cdot 10^5} = \unit[24,79]{l}  
\end{align*}


\section{Aufgabe 4}

Wir nehmen an, das Raumtemperatur $\unit[300]{K}$ und $\unit[1]{atm} = \unit[9,811 \cdot 10^4]{Pa}$:

\subsection*{a)}

Wir berechnen zunächst die molare Masse von $CO_2$:

\begin{align*}
M(CO_2) &= \unit[12]{g} + 2 \cdot \unit[16]{g} = \unit[44]{g}
\intertext{Damit erhalten wir unser $\nu$:}
\nu &= \frac{100}{44} = \unit[2,27]{mol/l}
\intertext{Nun können wir wieder in die ideale Gasgleichung einsetzen:}
V = \frac{\nu RT}{P} &= \frac{2,27 \cdot 8,134 \cdot 300}{9,811 \cdot 10^4} = \unit[57,8]{l}
\end{align*}


\subsection*{b)}

\begin{align*}
P &= \frac{\nu RT}{V} &= \frac{2,27 \cdot 8,134 \cdot 300}{80 \cdot 10^{-3}} = \unit[7,071 \cdot 10^4]{Pa}
\end{align*}


















\end{document}