\documentclass[11pt]{scrartcl}
\usepackage[T1]{fontenc}
\usepackage[a4paper, left=3cm, right=2cm, top=2cm, bottom=2cm]{geometry}
\usepackage[activate]{pdfcprot}
\usepackage[ngerman]{babel}
\usepackage[parfill]{parskip}
\usepackage[utf8]{inputenc}
\usepackage{kurier}
\usepackage{amsmath}
\usepackage{amssymb}
\usepackage{xcolor}
\usepackage{epstopdf}
\usepackage{txfonts}
\usepackage{fancyhdr}
\usepackage{graphicx}
\usepackage{prettyref}
\usepackage{hyperref}
\usepackage{eurosym}
\usepackage{setspace}
\usepackage{units}
\usepackage{eso-pic,graphicx}
\usepackage{icomma}
\usepackage{pdfpages}

\definecolor{darkblue}{rgb}{0,0,.5}
\hypersetup{pdftex=true, colorlinks=true, breaklinks=false, linkcolor=black, menucolor=black, pagecolor=black, urlcolor=darkblue}



\setlength{\columnsep}{2cm}


\newcommand{\arcsinh}{\mathrm{arcsinh}}
\newcommand{\asinh}{\mathrm{arcsinh}}
\newcommand{\ergebnis}{\textcolor{red}{\mathrm{Ergebnis}}}
\newcommand{\fehlt}{\textcolor{red}{Hier fehlen noch Inhalte.}}
\newcommand{\betanotice}{\textcolor{red}{Diese Aufgaben sind noch nicht in der Übung kontrolliert worden. Es sind lediglich meine Überlegungen und Lösungsansätze zu den Aufgaben. Es können Fehler enthalten sein!!! Das Dokument wird fortwährend aktualisiert und erst wenn das \textcolor{black}{beta} aus dem Dateinamen verschwindet ist es endgültig.}}
\newcommand{\half}{\frac{1}{2}}
\renewcommand{\d}{\, \mathrm d}
\newcommand{\punkte}{\textcolor{white}{xxxxx}}
\newcommand{\p}{\, \partial}
\newcommand{\dd}[1]{\item[#1] \hfill \\}

\renewcommand{\familydefault}{\sfdefault}
\renewcommand\thesection{}
\renewcommand\thesubsection{}
\renewcommand\thesubsubsection{}


\newcommand{\themodul}{Halbleiter und Nanotechnologie}
\newcommand{\thetutor}{Prof. Förster}
\newcommand{\theuebung}{Übung 3}

\pagestyle{fancy}
\fancyhead[L]{\footnotesize{C. Hansen}}
\chead{\thepage}
\rhead{}
\lfoot{}
\cfoot{}
\rfoot{}

\title{\themodul{}, \theuebung{}, \thetutor}


\author{Christoph Hansen \\ {\small \href{mailto:chris@university-material.de}{chris@university-material.de}} }

\date{}


\begin{document}

\maketitle

Dieser Text ist unter dieser \href{http://creativecommons.org/licenses/by-nc-sa/4.0/}{Creative Commons} Lizenz veröffentlicht.

\textcolor{red}{Ich erhebe keinen Anspruch auf Vollständigkeit oder Richtigkeit. Falls ihr Fehler findet oder etwas fehlt, dann meldet euch bitte über den Emailkontakt.}

\tableofcontents


\newpage



\section{Aufgabe 1}

Wir machen zunächst eine Definition. $v_w = \sqrt{\frac{2kT}{m}}$ ist die wahrscheinlichste Geschwindigkeit. Nun müssen wir den Integrationsoperator transformieren:

\begin{align*}
f(v) \d v &\rightarrow f(v') \d v' \\
\d v &= v_w \d v' \\ 
f(v) \d v &= \frac{4}{\sqrt{\pi}} \cdot \left( \frac{m}{2kT} \right)^{3/2} \cdot \left( \frac{v^2}{v_w^2} \right) v_w^2 \cdot e^{- \frac{v^2}{v_w^2}} \d v' \cdot v_w \\
&= \frac{4}{\sqrt{\pi}} \cdot v_w^3 \cdot \left( \frac{m}{2kT} \right)^{3/2} \cdot v'^2 \cdot e^{-v'^2} \d v' \\
&= \frac{4}{\sqrt{\pi}} \cdot \left( \frac{2kT}{m} \right)^{3/2} \cdot \left( \frac{m}{2kT} \right)^{3/2} \cdot v'^2 \cdot e^{-v'^2} \d v' \\
\Leftrightarrow f(v') \d v' &= \frac{4}{\sqrt{\pi}} \cdot v'^2 \cdot e^{-v'^2} \d v'
\end{align*}

\section{Aufgabe 2}

Die Wahrscheinlichkeit kann man über ein Integral berechnen:

\begin{align}
P &= \int_{v_1}^{v_2} f(v) \d v = \int_{400}^{410} \frac{4}{\sqrt{\pi}} \cdot \left( \frac{m}{2kT} \right)^{3/2} \cdot  v^2 \cdot e^{- \frac{m v^2}{2kT}} \d v
\intertext{Das sollte man mit einem geeignete Programm lösen. Man erhält dann:}
&\approx \unit[1,96]{\%}
\end{align}

\newpage

\section{Aufgabe 3}

\subsection*{a)}


Die mittlere Geschwindigkeit und die quadratisch gemittelte Geschwindigkeit sind so definiert:

\begin{align*}
<v> &= \sqrt{\frac{8 RT}{m \pi}} \qquad v_{RMS} = \sqrt{\frac{3 RT}{m}}
\intertext{Damit können wir jetzt die Geschwindigkeiten berechnen:}
\end{align*}


\begin{center}
	\centering
	\begin{tabular}{|l|l|l|l|l|}
		\hline
		& $H_2$ & $H_2O$ & $N_2$ & $CO_2$ \\ 
		\hline
		P/mbar  & $3 \cdot 10^{-10}$ & $10^{-10}$ & $1,5 \cdot 10^{-11}$ & $6 \cdot 10^{-12}$ \\ 
		\hline
		m  & 2 & 18 & 28 & 44 \\ 
		\hline
		$<v>$  & 1781 & 593 & 476 & 380 \\ 
		\hline
		$v_(RMS)$  & 1933 & 644 & 517 & 412 \\ 
		\hline 
	\end{tabular}
\end{center}
 


\subsection*{b)}

Die vorhandenen Moleküle zerlegen sich zu $H_2$, deshalb gibt es davon relativ viel.


\section{Aufgabe 4}

\begin{align*}
v_{RMS} = \sqrt{\frac{3 RT}{m}}
\end{align*}

Wenn die Geschwindigkeit auf das doppelte Steigen soll, dann müssen wir einen Faktor 4 in die Wurzel packen und haben damit einen Faktor 4 bei der Temperatur und habn dann statt Raumtemperatur $\unit[1200]{k}$. Das ist schon lecker warm....


\section{Aufgabe 5}

\begin{align*}
P V &= \nu RT
\end{align*}

Wenn der Druck verdoppelt wird, steigt die Temperatur auf das doppelte und damit wegen $E_{kin} = \frac{2}{3} kT$ die kinetische Energie ebenso. Bei verdoppeltem Volumen gilt das selbe.



\section{Aufgabe 6}


Hauptsächlich führen die Stöße der Moleküle untereinander zur Maxwell Verteilung. Die Energieverteilung ist dabei nach $E_{kin} = 3kT = \half m v^2$ nur von der Masse der Moleküle abhängig.

\section{Aufgabe 7}

Zunächst muss man wissen, das Wasser eine Molmasse von $M_{H_2O} = \unit[18]{g}$ hat. Dann können wir die beiden Volumina ausrechnen:

\begin{align*}
V_1 &= \frac{\nu RT}{P_1} = \frac{8,31 \cdot 288,15}{18 \cdot 100} = \unit[1,33]{m^3} \\
V_2 &=\frac{\nu RT}{P_2} =  \frac{8,31 \cdot 288,15}{18 \cdot 10^{-8}} = \unit[1,33 \cdot 10^{10}]{m^3}
\intertext{Die Dichte ist nun einfach Dichte pr Volumen:}
\rho_1 &= \frac{m}{V_1} = \frac{\unit[1]{g}}{\unit[1,33]{m^3}} = \unit[0,732]{g/m^3} \\
\rho_1 &= \frac{m}{V_1} = \frac{\unit[1]{g}}{\unit[1,33 \cdot 10^{10}]{m^3}} = \unit[0,732 \cdot 10^{-10}]{g/m^3}
\end{align*}

\section{Aufgabe 8}

Die Moleküldichte ist letztlich nicht anderes als die Anzahl Teilchen pro Volumen. Wir gehen von einer Temperatur von $\unit[300]{K}$ aus:

\begin{align*}
n_1 &= \frac{N}{V} = \frac{P_1}{kT} = \frac{10^5}{4,14 \cdot 10^{-21}} = \unit[2,4 \cdot 10^{25}]{m^3} \\
n_2 &= \frac{P_2}{kT} = \frac{10^5}{4,14 \cdot 10^{-21}} = \unit[2,4 \cdot 10^{22}]{m^3} \\
n_3 &= \frac{P_3}{kT} = \frac{10^5}{4,14 \cdot 10^{-21}} = \unit[2,4 \cdot 10^{16}]{m^3} \\
n_4 &= \frac{P_4}{kT} = \frac{10^5}{4,14 \cdot 10^{-21}} = \unit[2,4 \cdot 10^{11}]{m^3} 
\end{align*}


\newpage

\section{Aufgabe 9}

Wir betrachten den Massenfluss $q_m$:

\begin{align*}
q_m &= m \cdot \underbrace{\frac{\p N}{\p t}}_{q_n = j_n \cdot A} 
\intertext{Wir betrachten den Fluss allerdings unabhängig von der Fläche, deshalb erhalten wir:}
j_m &= m \cdot j_N = m \cdot n \cdot j_V = \frac{m \cdot n \cdot <v>}{4} \\
&= \frac{Mk}{R} \cdot \frac{n \cdot <v>}{4} = \frac{MP}{RT} \cdot \frac{n \cdot <v>}{4} \\
&= \frac{MP}{4 \cdot RT} \cdot \sqrt{\frac{8 \cdot RT}{M \cdot \pi}} = P \cdot \sqrt{\frac{M}{2 \cdot \pi \cdot RT}} = n \cdot \sqrt{\frac{mkT}{2 \cdot \pi}}
\end{align*}

Als Einheit haben wir dann $\left[ \frac{kg}{m^2s} \right]$.



\section{Aufgabe 10}

Energie ist zunächst:

\begin{align*}
E &= \half N m <v^2> 
\intertext{Leistung ist die Ableitung der Energie nach der Zeit:}
P &= \frac{\p E}{\p t} = \half \dot{N} m <v^2> = \half \underbrace{\frac{I}{kT}}_{I_T} m <v^2> = \frac{I}{kT} \frac{3}{2} \cdot kT = \frac{3}{2} I 
\intertext{Wir setzen ein:}
P &= 1,5 \cdot 10^{-6} \cdot 100 \cdot 10^{-3} = \unit[0,15]{\mu W}
\end{align*}

\textcolor{red}{Woher kommen die 100 und die 10E-3???? Klären!!!!}


\newpage


\section{Aufgabe 11}

Diese Aufgabe wurde nicht vollständig gelöst, es fehlen die Werte für $j_m$!

Wir schreiben uns zunächst die Formeln für die einzelnen Größen auf:

\begin{align*}
j_N &= \frac{n <v>}{4} = \frac{P <v>}{4kT} \\
j_m &= m \cdot j_N = \frac{Mk}{R} j_N = \frac{M}{N_A} j_N \\
j_V &= \sqrt{\frac{RT}{2 \pi M}} \\
q_N &= j_N \cdot A
\end{align*}

\begin{center}
	\begin{array}{c|c|c|c|c}
		& $H_2$ & $N_2$ & $H_2O$ & $CO_2$   \\ 
		\hline
	M/g	& 2 & 28 & 18 & 44 \\ 
		\hline
	<v>[m/s]	& 1782 & 476 & 594 & 380 \\ 
		\hline
	$j_N$	& $2,15 \cdot 10^{25}$ & $5,75 \cdot 10^{24}$ & $7,17 \cdot 10^{24}$ & $4,59 \cdot 10^{24}$ \\ 
		\hline
	$q_N$	& $2,15 \cdot 10^{19}$ & $5,75 \cdot 10^{18}$ & $7,17 \cdot 10^{18}$ & $4,59 \cdot 10^{18}$ \\ 
		\hline
	$j_V$	& 440 & 117 & 146 & 93
	\end{array} 
\end{center}


\end{document}