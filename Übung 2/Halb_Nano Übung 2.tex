\documentclass[11pt]{scrartcl}
\usepackage[T1]{fontenc}
\usepackage[a4paper, left=3cm, right=2cm, top=2cm, bottom=2cm]{geometry}
\usepackage[activate]{pdfcprot}
\usepackage[ngerman]{babel}
\usepackage[parfill]{parskip}
\usepackage[utf8]{inputenc}
\usepackage{kurier}
\usepackage{amsmath}
\usepackage{amssymb}
\usepackage{xcolor}
\usepackage{epstopdf}
\usepackage{txfonts}
\usepackage{fancyhdr}
\usepackage{graphicx}
\usepackage{prettyref}
\usepackage{hyperref}
\usepackage{eurosym}
\usepackage{setspace}
\usepackage{units}
\usepackage{eso-pic,graphicx}
\usepackage{icomma}
\usepackage{pdfpages}

\definecolor{darkblue}{rgb}{0,0,.5}
\hypersetup{pdftex=true, colorlinks=true, breaklinks=false, linkcolor=black, menucolor=black, pagecolor=black, urlcolor=darkblue}



\setlength{\columnsep}{2cm}


\newcommand{\arcsinh}{\mathrm{arcsinh}}
\newcommand{\asinh}{\mathrm{arcsinh}}
\newcommand{\ergebnis}{\textcolor{red}{\mathrm{Ergebnis}}}
\newcommand{\fehlt}{\textcolor{red}{Hier fehlen noch Inhalte.}}
\newcommand{\betanotice}{\textcolor{red}{Diese Aufgaben sind noch nicht in der Übung kontrolliert worden. Es sind lediglich meine Überlegungen und Lösungsansätze zu den Aufgaben. Es können Fehler enthalten sein!!! Das Dokument wird fortwährend aktualisiert und erst wenn das \textcolor{black}{beta} aus dem Dateinamen verschwindet ist es endgültig.}}
\newcommand{\half}{\frac{1}{2}}
\renewcommand{\d}{\, \mathrm d}
\newcommand{\punkte}{\textcolor{white}{xxxxx}}
\newcommand{\p}{\, \partial}
\newcommand{\dd}[1]{\item[#1] \hfill \\}

\renewcommand{\familydefault}{\sfdefault}
\renewcommand\thesection{}
\renewcommand\thesubsection{}
\renewcommand\thesubsubsection{}


\newcommand{\themodul}{Halbleiter und Nanotechnologie}
\newcommand{\thetutor}{Prof. Förster}
\newcommand{\theuebung}{Übung 3}

\pagestyle{fancy}
\fancyhead[L]{\footnotesize{C. Hansen}}
\chead{\thepage}
\rhead{}
\lfoot{}
\cfoot{}
\rfoot{}

\title{\themodul{}, \theuebung{}, \thetutor}


\author{Christoph Hansen \\ {\small \href{mailto:chris@university-material.de}{chris@university-material.de}} }

\date{}


\begin{document}

\maketitle

Dieser Text ist unter dieser \href{http://creativecommons.org/licenses/by-nc-sa/4.0/}{Creative Commons} Lizenz veröffentlicht.

\textcolor{red}{Ich erhebe keinen Anspruch auf Vollständigkeit oder Richtigkeit. Falls ihr Fehler findet oder etwas fehlt, dann meldet euch bitte über den Emailkontakt.}

\tableofcontents


\newpage



\section{Aufgabe 1}

Wir machen zunächst eine Definition. $v_w = \sqrt{\frac{2kT}{m}}$ ist die wahrscheinlichste Geschwindigkeit. Nun müssen wir den Integrationsoperator transformieren:

\begin{align*}
f(v) \d v &\rightarrow f(v') \d v' \\
\d v &= v_w \d v' \\ 
f(v) \d v &= \frac{4}{\sqrt{\pi}} \cdot \left( \frac{m}{2kT} \right)^{3/2} \cdot \left( \frac{v^2}{v_w^2} \right) v_w^2 \cdot e^{- \frac{v^2}{v_w^2}} \d v' \cdot v_w \\
&= \frac{4}{\sqrt{\pi}} \cdot v_w^3 \cdot \left( \frac{m}{2kT} \right)^{3/2} \cdot v'^2 \cdot e^{-v'^2} \d v' \\
&= \frac{4}{\sqrt{\pi}} \cdot \left( \frac{2kT}{m} \right)^{3/2} \cdot \left( \frac{m}{2kT} \right)^{3/2} \cdot v'^2 \cdot e^{-v'^2} \d v' \\
\Leftrightarrow f(v') \d v' &= \frac{4}{\sqrt{\pi}} \cdot v'^2 \cdot e^{-v'^2} \d v'
\end{align*}

\section{Aufgabe 2}

Die Wahrscheinlichkeit kann man über ein Integral berechnen:

\begin{align}
P &= \int_{v_1}^{v_2} f(v) \d v = \int_{400}^{410} \frac{4}{\sqrt{\pi}} \cdot \left( \frac{m}{2kT} \right)^{3/2} \cdot  v^2 \cdot e^{- \frac{m v^2}{2kT}} \d v
\intertext{Das sollte man mit einem geeignete Programm lösen. Man erhält dann:}
&\approx \unit[1,96]{\%}
\end{align}

\newpage

\section{Aufgabe 3}

\subsection*{a)}


Die mittlere Geschwindigkeit und die quadratisch gemittelte Geschwindigkeit sind so definiert:

\begin{align*}
<v> &= \sqrt{\frac{8 RT}{m \pi}} \qquad v_{RMS} = \sqrt{\frac{3 RT}{m}}
\intertext{Damit können wir jetzt die Geschwindigkeiten berechnen:}
\end{align*}


\begin{center}
	\centering
	\begin{tabular}{|l|l|l|l|l|}
		\hline
		& $H_2$ & $H_2O$ & $N_2$ & $CO_2$ \\ 
		\hline
		P/mbar  & $3 \cdot 10^{-10}$ & $10^{-10}$ & $1,5 \cdot 10^{-11}$ & $6 \cdot 10^{-12}$ \\ 
		\hline
		m  & 2 & 18 & 28 & 44 \\ 
		\hline
		$<v>$  & 1781 & 593 & 476 & 380 \\ 
		\hline
		$v_(RMS)$  & 1933 & 644 & 517 & 412 \\ 
		\hline 
	\end{tabular}
\end{center}
 


\subsection*{b)}

Die vorhandenen Moleküle zerlegen sich zu $H_2$, deshalb gibt es davon relativ viel.


\section{Aufgabe 4}

\begin{align*}
v_{RMS} = \sqrt{\frac{3 RT}{m}}
\end{align*}

Wenn die Geschwindigkeit auf das doppelte Steigen soll, dann müssen wir einen Faktor 4 in die Wurzel packen und haben damit einen Faktor 4 bei der Temperatur und habn dann statt Raumtemperatur $\unit[1200]{k}$. Das ist schon lecker warm....


\section{Aufgabe 5}

\begin{align*}
P V &= \nu RT
\end{align*}

Wenn der Druck verdoppelt wird, steigt die Temperatur auf das doppelte und damit wegen $E_{kin} = \frac{2}{3} kT$ die kinetische Energie ebenso. Bei verdoppeltem Volumen gilt das selbe.



\section{Aufgabe 6}


Hauptsächlich führen die Stöße der Moleküle untereinander zur Maxwell Verteilung. Die Energieverteilung ist dabei nach $E_{kin} = 3kT = \half m v^2$ nur von der Masse der Moleküle abhängig.


\section{Aufgabe 7}

Zunächst muss man wissen, das Wasser eine Molmasse von $M_{H_2O} = \unit[18]{g}$ hat. Dann können wir die beiden Volumina ausrechnen:

\begin{align*}
V_1 &= \frac{\nu RT}{P_1} = \frac{8,31 \cdot 288,15}{18 \cdot 100} = \unit[1,33]{m^3} \\
V_2 &=\frac{\nu RT}{P_2} =  \frac{8,31 \cdot 288,15}{18 \cdot 10^{-8}} = \unit[1,33 \cdot 10^{10}]{m^3}
\intertext{Die Dichte ist nun einfach Dichte pr Volumen:}
\rho_1 &= \frac{m}{V_1} = \frac{\unit[1]{g}}{\unit[1,33]{m^3}} = \unit[0,732]{g/m^3} \\
\rho_1 &= \frac{m}{V_1} = \frac{\unit[1]{g}}{\unit[1,33 \cdot 10^{10}]{m^3}} = \unit[0,732 \cdot 10^{-10}]{g/m^3}
\end{align*}

\section{Aufgabe 8}

Die Moleküldichte ist letztlich nicht anderes als die Anzahl Teilchen pro Volumen. Wir gehen von einer Temperatur von $\unit[300]{K}$ aus:

\begin{align*}
n_1 &= \frac{N}{V} = \frac{P_1}{kT} = \frac{10^5}{4,14 \cdot 10^{-21}} = \unit[2,4 \cdot 10^{25}]{m^3} \\
n_2 &= \frac{P_2}{kT} = \frac{10^5}{4,14 \cdot 10^{-21}} = \unit[2,4 \cdot 10^{22}]{m^3} \\
n_3 &= \frac{P_3}{kT} = \frac{10^5}{4,14 \cdot 10^{-21}} = \unit[2,4 \cdot 10^{16}]{m^3} \\
n_4 &= \frac{P_4}{kT} = \frac{10^5}{4,14 \cdot 10^{-21}} = \unit[2,4 \cdot 10^{11}]{m^3} 
\end{align*}


\newpage

\section{Aufgabe 9}

Wir betrachten den Massenfluss $q_m$:

\begin{align*}
q_m &= m \cdot \underbrace{\frac{\p N}{\p t}}_{q_n = j_n \cdot A} 
\intertext{Wir betrachten den Fluss allerdings unabhängig von der Fläche, deshalb erhalten wir:}
j_m &= m \cdot j_N = m \cdot n \cdot j_V = \frac{m \cdot n \cdot <v>}{4} \\
&= \frac{Mk}{R} \cdot \frac{n \cdot <v>}{4} = \frac{MP}{RT} \cdot \frac{n \cdot <v>}{4} \\
&= \frac{MP}{4 \cdot RT} \cdot \sqrt{\frac{8 \cdot RT}{M \cdot \pi}} = P \cdot \sqrt{\frac{M}{2 \cdot \pi \cdot RT}} = n \cdot \sqrt{\frac{mkT}{2 \cdot \pi}}
\end{align*}

Als Einheit haben wir dann $\left[ \frac{kg}{m^2s} \right]$.



\section{Aufgabe 10}

Energie ist zunächst:

\begin{align*}
E &= \half N m <v^2> 
\intertext{Leistung ist die Ableitung der Energie nach der Zeit:}
P &= \frac{\p E}{\p t} = \half \dot{N} m <v^2> = \half \underbrace{\frac{I}{kT}}_{I_T} m <v^2> = \frac{I}{kT} \frac{3}{2} \cdot kT = \frac{3}{2} I 
\intertext{Wir setzen ein:}
P &= 1,5 \cdot 10^{-6} \cdot 100 \cdot 10^{-3} = \unit[0,15]{\mu W}
\end{align*}

\textcolor{red}{Woher kommen die 100 und die 10E-3???? Klären!!!!}


\newpage


\section{Aufgabe 11}

Diese Aufgabe wurde nicht vollständig gelöst, es fehlen die Werte für $j_m$!

Wir schreiben uns zunächst die Formeln für die einzelnen Größen auf:

\begin{align*}
j_N &= \frac{n <v>}{4} = \frac{P <v>}{4kT} \\
j_m &= m \cdot j_N = \frac{Mk}{R} j_N = \frac{M}{N_A} j_N \\
j_V &= \sqrt{\frac{RT}{2 \pi M}} \\
q_N &= j_N \cdot A
\end{align*}

\begin{center}
	\begin{tabular}{c|c|c|c|c}
		& $H_2$ & $N_2$ & $H_2O$ & $CO_2$   \\ 
		\hline
	M/g	& 2 & 28 & 18 & 44 \\ 
		\hline
	<v>[m/s]	& 1782 & 476 & 594 & 380 \\ 
		\hline
	$j_N$	& $2,15 \cdot 10^{25}$ & $5,75 \cdot 10^{24}$ & $7,17 \cdot 10^{24}$ & $4,59 \cdot 10^{24}$ \\ 
		\hline
	$q_N$	& $2,15 \cdot 10^{19}$ & $5,75 \cdot 10^{18}$ & $7,17 \cdot 10^{18}$ & $4,59 \cdot 10^{18}$ \\ 
		\hline
	$j_V$	& 440 & 117 & 146 & 93
	\end{tabular} 
\end{center}



\section{Aufgabe 12}

\subsection*{a)}

Der Leitwert is allgemein:

\begin{align*}
C &= \sqrt{\frac{8RT}{M \pi}} \cdot \frac{A}{4}
\intertext{Bis auf die Molmasse ist alles Konstanten, also können wir schreiben:}
C_{N_2} &= \frac{1}{\sqrt{M_{N_2}}} \cdot a \cdot A 
\intertext{Für Wasserstoff gilt dann:}
C_{H_2} &= \frac{1}{\sqrt{M_{H_2}}} \cdot a \cdot A 
\intertext{Es ergibt sich dann:}
C_{H_2} &= C_{N_2} \cdot \sqrt{\frac{28}{2}} = \unit[3,77]{ml/s}
\end{align*}


\subsection*{b)}

Die Funktion für die Druck ist:

\begin{align*}
P &= P_0 \cdot e^{-\frac{t}{\tau}} \qquad \text{mit} \qquad \tau = \frac{V}{S_{eff}}
\intertext{Wir bestimmen die jeweiligen $\tau$:}
\tau_{H_2} &= \frac{2}{3,77 \cdot 10^{-3}} = \unit[534,8]{s} \\
\tau_{N_2} &= \frac{2}{10^{-3}} = \unit[2000]{s}
\intertext{Daraus ergeben sich die Partialdrücke:}
P_{H_2} &= 10^{-3} \cdot e^{- \frac{1800}{534,8}} = \unit[3,45 \cdot 10^{-5}]{mbar} \\
P_{N_2} &= 10^{-3} \cdot e^{- \frac{1800}{2000}} = \unit[4 \cdot 10^{-4}]{mbar}
\intertext{Der Gesamtdruck ergibt sich nun aus der Addition der beiden Drücke:}
P_{ges} &= P_{H_2} + P_{N_2} = \unit[4,345 \cdot 10^{-4}]{mbar}
\end{align*}



\section{Aufgabe 13}

\subsection*{a)}

\begin{align*}
L &= Pa \cdot q_V \qquad \text{mit} \qquad  Pa = \text{Außendruck}, q_V = \text{Volumenfluss} \\
q_{N_2} &= \frac{L}{Pa} = \frac{10^{-10}}{10^5} = \unit[10^{-15}]{m^3/s} \\
q_{He} &= \sqrt{\frac{28}{4}} \cdot q_{N_2} = \unit[2,6 \cdot 10^{-15}]{m^3/s}
\end{align*}


\subsection*{b)}

\begin{align*}
q_N &= n \cdot q_V = \frac{P q_V}{kT} = \frac{L}{kT} = \frac{10^{-10}}{1,38 \cdot 10^{-23} \cdot 300} = \unit[2,41 \cdot 10^{10}]{Teilchen/s}
\end{align*}


\subsection*{c)}

Wir berechnen die Größe den Wafers:

\begin{align*}
A &= \frac{d^2 \pi}{4} = \frac{50 \cdot 10^{-3} \cdot \pi}{4} = \unit[1,96 \cdot 10^{-3}]{m^2}
\intertext{Wir bestimmen wie viele Teilchen pro $m^2$ auftreffen:}
j_N &= \frac{2,41 \cdot 10^{10}}{1,96 \cdot 10^{-3}} = \unit[1,22 \cdot 10^9]{Teilchen/m^2 s}
\intertext{Nun bestimmen wir die relative Haufigkeit einer Dotierung:}
\frac{1,22 \cdot 10^{6}}{6,8 \cdot 10^{14}} &= 1,8 \cdot 10^{-9}
\end{align*}

Das heißt, das jedes Milliarste Atom ein Fehlatom ist. Damit ist die Verschmutzung tolerierbar. Genau ist die Dotierung:

\begin{align*}
\left( \sqrt{6,8 \cdot 10^{14}} \right)^3 &= \unit[1,7 \cdot 10^{22}]{Teilchen/cm^3} 
\intertext{Die Dotierung ist dann:}
1,7 \cdot 10^{22} \cdot 1,8 \cdot 10^{-9} &= \unit[3 \cdot 10^{13}]{Teilchen/cm^3}
\end{align*}



\subsection*{d)}


Die Leitfähigkeit ist:

\begin{align*}
\sigma &= n \cdot e \cdot \mu = 3 \cdot 10^{13} \cdot 1,6 \cdot 10^{-19} \cdot 4 = \unit[1,9 \cdot 10^{-5}]{1/ \Omega m}
\end{align*}


\subsection*{e)}

Der Widerstand berechnen sich aus der Leitfähigkeit und der Fläche:

\begin{align*}
R &= \frac{l}{\sigma \cdot A} = \frac{10^{-6}}{1,9 \cdot 10^{-5} \cdot 10^{-4}} = \unit[526,31]{\Omega}
\end{align*}




























\end{document}