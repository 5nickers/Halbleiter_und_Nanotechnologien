\input{header.tex}


\begin{document}

\maketitle

Dieser Text ist unter dieser \href{http://creativecommons.org/licenses/by-nc-sa/4.0/}{Creative Commons} Lizenz veröffentlicht.

\textcolor{red}{Ich erhebe keinen Anspruch auf Vollständigkeit oder Richtigkeit. Falls ihr Fehler findet oder etwas fehlt, dann meldet euch bitte über den Emailkontakt.}

\tableofcontents


\newpage



\section{Aufgabe 1}

\subsection*{a)}

Der Leitwert ist auch der Volumenfluss $q_V = C$, also:

\begin{align*}
C_{H_2} &= \sqrt{\frac{RT}{2 \pi M}} \cdot A = \sqrt{\frac{8,31 \cdot 300}{2 \pi \cdot 2 \cdot 10^{-3}}} \cdot \frac{\pi \cdot \left( 0,3 \cdot 10^{-6} \right)^2}{4} = \unit[3,14 \cdot 10^{-11}]{m^3/s} \\
C_{N_2} &= \sqrt{\frac{RT}{2 \pi M}} \cdot A = \sqrt{\frac{8,31 \cdot 300}{2 \pi \cdot 28 \cdot 10^{-3}}} \cdot \frac{\pi \cdot \left( 0,3 \cdot 10^{-6} \right)^2}{4} = \unit[8,4 \cdot 10^{-12}]{m^3/s}
\end{align*}


\subsection*{b)}

Die Leckraten sind:

\begin{align*}
L_{H_2} &= P_a \cdot q_V = 0,5 \cdot 10^5 \cdot 3,14 \cdot 10^{-11} = \unit[1,5 \cdot 10^{-6}]{Pa \ m^3 /s} \\
L_{N_2} &= 0,5 \cdot 10^5 \cdot 8,4 \cdot 10^{-12} = \unit[2,5 \cdot 10^{-7}]{Pa \ m^3 /s}
\end{align*}


\subsection*{c)}


Wir haben einen linearen Druckanstieg. $I_E$ ist der einlaufende Gasstrom also die Leckrate:

\begin{align*}
P &= P_0 + \frac{I_E}{V_r} \cdot t \\
P_{H_2} &= \frac{1,5 \cdot 10^{-8}}{0,3} \cdot 600 = \unit[3 \cdot 10^{-5}]{mbar} \\
P_{H_2} &= \frac{2,5 \cdot 10^{-9}}{0,3} \cdot 600 = \unit[5 \cdot 10^{-6}]{mbar} \\
\hfil \\
P_{ges} &= P_0 + P_{N_2} + P_{H_2} = \unit[3,5 \cdot 10^{-5}]{mbar}
\end{align*}


\subsection*{d)}

Aus den Enddrücken von Wasserstoff und Stickstoff berechnen wir den Gesamtenddruck:

\begin{align*}
P_{e,H_2} &= \frac{P_a \cdot C}{S} = \frac{0,5 \cdot 10^5 \cdot 3,1 \cdot 10^{-11}}{0,2} = \unit[7,75 \cdot 10^{-6}]{Pa} \\
P_{e,N_2} &= \frac{P_a \cdot C}{S} = \frac{0,5 \cdot 10^5 \cdot 8,9 \cdot 10^{-12}}{0,2} = \unit[1,26 \cdot 10^{-6}]{Pa} \\
\hfil \\
P_{ges} &= P_{e,H_2} + P_{e,N_2} = \unit[9 \cdot 10^{-6}]{Pa}
\end{align*}



\section{Aufgabe 2}

\subsection*{a)}

Der Leitwert des Rohres ist:

\begin{align*}
C_L &= 1,2 \cdot \sqrt{\frac{T}{M}} \cdot \frac{D^3}{L} = 1,2 \cdot \sqrt{\frac{300}{28 \cdot 10^{-3}}} \cdot \frac{\left( 50 \cdot 10^{-3} \right)^3}{5} = \unit[3,1 \cdot 10^{-3}]{m^3/s} = \unit[3,1]{l/s}
\intertext{Den Klausingfaktor müssen wir nicht berücksichtigen, das das Verhältnis vonm Länge zu Durchmesser sehr groß ist.}
\end{align*}


\subsection*{b)}

\begin{align*}
\frac{1}{S_{eff}} &= \frac{1}{C_V} + \frac{1}{C_L} + \frac{1}{S_P} = \frac{1}{10} + \frac{1}{3,1} + \frac{1}{200} = \unit[0,43]{s/l} \\
\Leftrightarrow S_{eff} &= \unit[2,2]{l/s}
\end{align*}


\section{Aufgabe 3}

Das Rohr ist $\unit[3]{m}$ lang!


\subsection*{a)}

Die Knudsenzahl lässt sich so bestimmen:

\begin{align*}
K &= \frac{\lambda}{D} \qquad \text{mit} \qquad \lambda = \frac{\lambda_P}{\bar{P}} \\
&= \frac{\frac{\lambda_P}{\frac{P_1 + P_2}{2}}}{D} = \frac{\frac{79 \cdot 10^{-6}}{275}}{15 \cdot 10^{-3}} = 1,9 \cdot 10^{-3} < 10^{-2}
\end{align*}

Es handelt sich um eine reibungsbehaftete Strömung.


\subsection*{b)}

\begin{align*}
C_{Rohr} &= 2,454 \cdot 10^{-2} \cdot \frac{D^4}{\eta l} \cdot \bar{P} = 2,454 \cdot 10^{-2} \cdot \frac{\left( 1,5 \cdot 10^{-3} \right)^4 \cdot 275}{0,0086 \cdot 10^{-3} \cdot 3} = \unit[1,32 \cdot 10^{-2}]{m^3/s}
\end{align*}


\subsection*{c)}

\begin{align*}
I &= C \cdot \Delta P = 1,32 \cdot 10^{-2} \cdot \left( 300 - 250 \right) = \unit[6,6 \cdot 10^{-1}]{Pa \ m^3 / s}
\end{align*}



\section{Aufgabe 4}

\subsection*{a)}

Wir müssen die Formel für die Energie zweifach nach $k$ ableiten:

\begin{align*}
\frac{\p E}{\p k} &= 2 \cdot 9,13 \cdot 10^{-38} \cdot \left( k - k_0 \right) \\
\frac{\p^2 E}{\p^2 k} &= 2 \cdot 9,13 \cdot 10^{-38} 
\intertext{Nun müssel wir noch die masseabhängige Energieformel ableiten:}
E &= \frac{\hbar^2 \cdot k^2}{2m} \\
\frac{\p E}{\p k} &= \frac{\hbar^2 \cdot k}{m} \\
\frac{\p^2 E}{\p^2 k} &= \frac{\hbar^2}{m^*}
\intertext{Nun können wir gleichsetzen:}
\Leftrightarrow \frac{1}{m^*} &= \frac{\p^2 E}{\p^2 k} \cdot \frac{1}{\hbar^2} = \frac{2 \cdot 9,13 \cdot 10^{-38}}{\left( 1,054 \cdot 10^{-34} \right)^2} = 1,6 \cdot 10^{31} \\
\Leftrightarrow m^* &= \unit[6,08 \cdot 10^{-32}]{kg} \\
\Rightarrow \frac{m^*}{m_e} &= \frac{6,08 \cdot 10^{-32}}{9,1 \cdot 10^{-31}} = 0,067
\end{align*}


\subsection*{b)}


\begin{figure}[h]
	\centering
	\includegraphics[scale=0.1]{A4_1.jpg}
\end{figure}

Es ist ein indirekter Halbleiter, das es den Versatz $\Delta k$ gibt.



\subsection*{c)}

\begin{align*}
n &= N_c \cdot e^{- \frac{E_L - E_F}{kT}} \\
\Leftrightarrow \frac{n}{N_c} &= e^{- \frac{\Delta E}{kT}} \\
\Leftrightarrow \frac{\Delta E}{kT} &= \ln \left( \frac{n}{N_c} \right) = \ln \left( \frac{10^{23}}{4,35 \cdot 10^{23}} \right) = 1,47
\intertext{Wir rechnen mit $kT = \unit[26]{meV}$}
\Leftrightarrow \Delta E &= 1,47 \cdot 26 = \unit[38,2]{meV}
\end{align*}



\section{Aufgabe 5}

\subsection*{a)}

\begin{figure}[h]
	\centering
	\includegraphics[scale=0.16]{A5_1.jpg}
\end{figure}



\subsection*{b)}

\begin{align*}
\phi_{SB} &= V_{bi} + \phi_n \\
\Leftrightarrow V_{bi} &= \phi_{SB} - \phi_n = 0,65 - 0,09 = \unit[0,56]{V}
\end{align*}


\subsection*{c)}

\begin{align*}
w &= \sqrt{\frac{2 \epsilon_r \epsilon_0 \cdot \left( V_{bi} + V_R \right)}{e \cdot N_d}} = \sqrt{\frac{2 \epsilon_r \epsilon_0 \cdot 0,56}{1,6 \cdot 10^{-19} \cdot 1,6 \cdot 10^{22}}} = \unit[220]{nm}
\end{align*}


\subsection*{d)}

\begin{align*}
C' &= \frac{\epsilon_r \cdot \epsilon_0}{w} = \frac{8,85 \cdot 10^{-12} \cdot 12,9}{220 \cdot 10^{-9}} = \unit[5,1 \cdot 10^{-4}]{F/m}
\end{align*}


\section{Aufgabe 6}


Wir müssen jeweils zwei Lastgeraden berechnen. Diese gehen laufen von $\unit[5]{V}$ auf der x-Achse zu dem berechneten Strom auf der y-Achse:

\begin{align*}
I_{10k} &= \frac{U_B}{R_L} = \frac{5}{10000} = \unit[0,5]{mA} \\
I_{20k} &= \frac{U_B}{R_L} = \frac{5}{20000} = \unit[0,25]{mA}
\end{align*}



\subsection*{b)}

\begin{center}
	\begin{tabular}{c|c|c}
	$U_i$	& $U_{out}$ 10k & $U_{out}$ 20k \\ 
		\hline  
	0	& 5 & 5 \\ 
		\hline 
	5	& 2,5 & 1 \\ 
	\end{tabular} 
\end{center}


Man würde den $\unit[20]{k \Omega}$ Widerstand nehmen, da dieser mehr schwankt.















\end{document}