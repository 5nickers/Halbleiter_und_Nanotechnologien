\input{header.tex}


\begin{document}

\maketitle

Dieser Text ist unter dieser \href{http://creativecommons.org/licenses/by-nc-sa/4.0/}{Creative Commons} Lizenz veröffentlicht.

\textcolor{red}{Ich erhebe keinen Anspruch auf Vollständigkeit oder Richtigkeit. Falls ihr Fehler findet oder etwas fehlt, dann meldet euch bitte über den Emailkontakt.}

\tableofcontents


\newpage



\section{Aufgabe 1}

\subsection*{a)}

\begin{align*}
C' &= \frac{\epsilon_0 \epsilon_R}{w} \\
\Leftrightarrow w &= \frac{\epsilon_0 \epsilon_R}{C'} = \frac{12,9 \cdot 8,85 \cdot 10^{-12}}{10^{-3}} = \unit[114]{nm}
\end{align*}


\subsection*{b)}

\begin{align*}
w &= \sqrt{\frac{2 \epsilon_r \cdot \epsilon_0 \cdot \left( V_{bi} + V_R \right)}{e N_d}} \\
\Leftrightarrow V_{bi} + V_R &= \frac{w^2 e N_d}{2 \epsilon_r \epsilon_0} \\
\Leftrightarrow V_R &= -0,6 + \frac{w^2 e N_d}{2 \epsilon_r \epsilon_0} \\
&= -0,6 + \frac{\left( 114 \cdot 10^{-19} \right)^2 \cdot 1,6 \cdot 10^{-19} \cdot 10^{23}}{12,9 \cdot 8,85 \cdot 10^{-12}} = \unit[0,31]{V}
\end{align*}


\subsection*{c)}

Hier gilt $n = N_d$. Zur Anschaulichkeit ein Bild:

\begin{figure}[h]
	\centering
	\includegraphics[scale=0.15]{A1_1.jpg}
\end{figure}


\begin{align*}
n &= N_c \cdot e^{- \frac{E_c - E_F}{kT}} \\
\Leftrightarrow \ln\left( \frac{n}{N_c} \right) &= - \frac{\phi_n e}{kT} \\
\Leftrightarrow \phi_n &= \frac{kT}{e} \cdot \ln \left( \frac{N_c}{n} \right) = \unit[38]{mV}
\intertext{Damit ist die Schottky Barriere:}
\phi_{SB} &= 0,038 + 0,6 = \unit[0,638]{mV}
\end{align*}



\section{Aufgabe 2}

\subsection*{a)}

\begin{align*}
j_s &= A^* T^2 \cdot e^{-e \cdot \frac{\phi_{SB} - \Delta \phi}{kT}}
\intertext{Wir vernachlässigen allerdings $\Delta \phi$, weil wir das E-Feld nicht kennen.}
&= 260 \cdot 300^2 \cdot e^{- \frac{0,72}{0,026}} = \unit[2,2 \cdot 10^{-5}]{A/cm^2}
\end{align*}

\subsection*{b)}

\begin{align*}
I_S &= j_s \cdot A = 2,2 \cdot 10^{-5} \cdot 10^{-6} = \unit[22 \cdot 10^{-12}]{A}
\end{align*}


\subsection*{c)}

Diese Teilaufgabe ist eigentlich falsch, da wie den Serienwiderstand nicht berücksichtigen!

\begin{align*}
I &= I_S \cdot \left( e^{\frac{U}{U_T}} - 1 \right) \\
\frac{U}{U_T} &= \ln \left( \frac{I}{I_s} + 1 \right) \\
\Leftrightarrow U &= U_T \cdot \ln \left( \frac{20 \cdot 10^{-3}}{22 \cdot 10^{-12}} + 1 \right) = \unit[0,52]{V}
\end{align*}

\newpage

\section{Aufgabe 3}

\subsection*{a)}

\begin{figure}[h]
	\centering
	\includegraphics[scale=0.15]{A3_1.jpg}
\end{figure}

\newpage

\subsection*{b)}


\begin{figure}[h]
	\centering
	\includegraphics[scale=0.13]{A3_2.jpg}
	\caption{n-Typ}
\end{figure}


\begin{figure}[h]
	\centering
	\includegraphics[scale=0.13]{A3_3.jpg}
	\caption{p-Typ}
\end{figure}

\newpage

\subsection*{c)}

Wir rechnen anders als in der Aufgabenstellung mit $V_{bi} = \unit[0,6]{V}$

\begin{figure}[h]
	\centering
	\includegraphics[scale=0.13]{A3_4.jpg}
\end{figure}


\subsubsection{c1)}


\begin{align*}
w &= \sqrt{\frac{2 \epsilon_r \cdot \epsilon_0 \cdot \left( V_{bi} + V_R \right)}{e N_d}} = \sqrt{\frac{2 \cdot 8,85 \cdot 10^{-12} \cdot 12,9 \cdot 0,6}{1,6 \cdot 10^{-19} \cdot 10^{23}}} = \unit[93]{nm}
\end{align*}


\subsubsection{c2)}

\begin{align*}
w &= \sqrt{\frac{2 \epsilon_r \cdot \epsilon_0 \cdot \left( V_{bi} + V_R \right)}{e N_d}} \\
\Leftrightarrow V_R &= \frac{\left( 300 \cdot 10^{-9} \right)^2 \cdot 1,6 \cdot 10^{-19} \cdot 10^{23}}{2 \cdot 12,9 \cdot 8,85 \cdot 10^{-12}} = \unit[5,7]{V}
\end{align*}

\subsection*{d)}

Wir berechnen die Dicken für die einzelnen Reversespannungen:

\begin{align*}
w(0) &= \unit[93]{nm} \\
w(1) &= \unit[152]{nm} \\
w(3) &= \unit[227]{nm} 
\intertext{Den Widerstand bestimmen wir dann so:}
R &= \frac{l \rho}{A} = \frac{l}{ne \mu A}
\end{align*}

\newpage

Anschaulich sieht das ungefähr so aus:

\begin{figure}[h]
	\centering
	\includegraphics[scale=0.1]{A3_5.jpg}
\end{figure}

Die Widerstände ergeben sich dann zu:

\begin{align*}
R(0) &= \frac{10^{-6}}{10^{23} \cdot 1,6 \cdot 10^{-19} \cdot 0,2 \cdot \left( 300 - 93 \right) \cdot 10^{-9} \cdot 10^{-5}} = \unit[151]{\Omega} \\
R(1) &= \frac{10^{-6}}{10^{23} \cdot 1,6 \cdot 10^{-19} \cdot 0,2 \cdot \left( 300 - 152 \right) \cdot 10^{-9} \cdot 10^{-5}} = \unit[211]{\Omega} \\
R(3) &= \frac{10^{-6}}{10^{23} \cdot 1,6 \cdot 10^{-19} \cdot 0,2 \cdot \left( 300 - 227 \right) \cdot 10^{-9} \cdot 10^{-5}} = \unit[428]{\Omega} 
\intertext{Die Leitwerte sind dann:}
G(0) &= \unit[6,6 \cdot 10^{-3}]{1/\Omega} \\
G(1) &= \unit[4,74 \cdot 10^{-3}]{1/\Omega}\\
G(3) &= \unit[2,33 \cdot 10^{-3}]{1/\Omega}
\end{align*}

\newpage


\section{Aufgabe 4}

\subsection*{a)}

\begin{figure}[h]
	\centering
	\includegraphics[scale=0.16]{A4_1.jpg}
\end{figure}


\subsection*{b)}

Die Raumladungszone wird nach rechts größer.


\subsection*{c)}

\begin{figure}[h]
	\centering
	\includegraphics[scale=0.12]{A4_2.jpg}
\end{figure}



\section{Aufgabe 5}

\subsection*{a)}

\begin{figure}[h]
	\centering
	\includegraphics[scale=0.1]{A5_1.jpg}
\end{figure}

Es gilt das $2 R_C$, wenn $l_i = 0$

\begin{align*}
2 R_C &= 0,516 \\
\Leftrightarrow R_C &= \unit[0,258]{\Omega}
\end{align*}


\subsection*{b)}

Wir müssen zunächst $l_0$ bestimmen, dasmit wir dann $l_T$ berechnen können. $l_0$ ist der Schnittpunkt bei $R = 0$:

\begin{align*}
0 &= 0,516 + 0,096 \cdot l_0 \\
\Leftrightarrow l_0 &= - \frac{0,516}{0,096} = \unit[- 5,4]{\mu m}
\intertext{Wir setzen $r_s = R_s$ vorraus:}
l_0 &= 2 \cdot \frac{R_s}{r_s} \cdot l_T \approx 2 \cdot l_T \\
\Leftrightarrow l_T &= \frac{l_0}{2} = \unit[-2,7]{\mu m}
\intertext{Nun bestimmen wir noch die effektice Kontaktfläche:}
A_{eff} &= 10 \cdot 10^{-6} \cdot l_T = 10 \cdot 10^{-6} \cdot 2,7 \cdot 10^{-6} = \unit[27 \cdot 10^{-12}]{m^2}
\end{align*}


\subsection*{d)}

\begin{align*}
\rho_i &= R_c \cdot A = 0,258 \cdot 27 \cdot 10^{-12} = \unit[6,69 \cdot 10^{-12}]{\Omega \ m^2}
\end{align*}


\subsection*{e)}

\begin{align*}
R &= \frac{\rho \cdot e}{A} = \frac{6,69 \cdot 10^{-12}}{4 \cdot 0,5 \cdot 10^{-12}} = \unit[3,5]{\Omega}
\end{align*}





























\end{document}